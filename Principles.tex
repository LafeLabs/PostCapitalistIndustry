\subsection{Statement of Principles}\label{statement-of-principles}

\begin{itemize}
\tightlist
\item
  All technology should be free
\item
  All people should be free to leave a technical sphere and enter or
  build another one
\item
  All national borders are not legitimate and must be abolished
\item
  The world is magical. The properties we have always called ``magic''
  can be ascribed to all things in the physical world, and these powers
  can be harnessed by the techniques of Trash Magic
\item
  Capitalism cannot and should not be reformed, it should be opposed in
  all places and times until it dies
\item
  The concept of professionalism is harmful to the human condition, it
  poisons the soul, and is evil\\
\item
  The concept of finite number to represent human values is a mind virus
  that must be purged. The infinite exposes deeper truths than the
  finite. These problems go to the deepest level of our mathematical
  thought from arithmetic to the underlying axioms of mathematics
\item
  Morality consists of a set of axioms. An axiom is a unproven statement
  which we take to be true in order to build up a system of thought
  which can guide action. The principles in this list are put forth as
  axioms.
\item
  It is not our role to debate capitalism with its defenders. Every
  possible basic argument for or against capitalism already exists on
  the Internet. Our job is to build a set of moral axioms, a set of
  technical skills and knowledge and build up a practical society from
  that. It is not our job to waste time repeating the same arguments
  with capitalist apologists and time wasters.
\item
  No technology should be made from mass-mined materials
\item
  The sum total of all money that exists in the world is a small
  fraction of what would be needed to compensate the victims of
  capitalism from its crimes(e.g.~slavery and imperialism), thus there
  can be no justice within that system
\item
  Every single word said every single idea ever put forth by an
  economist is a vicious lie. Economics is not a science, and this work
  is rejecting traditional science anyway. It is not our job to argue
  with the economist it is our job to build a better world in which they
  are not welcome.
\item
  The wage system must be abolished
\item
  End work. I am against work in all forms. We must attack the concept
  of work at all levels.
\item
  Technology is personal, as it should be. Relationships between
  technology and the human body are always in mind.
\end{itemize}

\subsection{Design Rules}\label{design-rules}

Engineers who build technology usually use something called ``design
rules'' and ``figures of merit'' as guides for how to build a thing. The
following are the different design rules in which we may deviate from
capitalism to end up with technically different results:

\begin{enumerate}
\def\labelenumi{\arabic{enumi}.}
\tightlist
\item
  The more general solution is always better
\item
  The Most readily available materials are always the first choice to
  use as well as to study
\item
  The most obvious solution is the best, although what is obvious may
  not be obvious
\item
  Self similarity is a desirable property, and by default it will be
  built in for several(but not infinite!) zoom factors to all technical
  systems
\item
  All technology is art, all art is technology
\item
  All technology contains its own data, is linked to itself on the web,
  self documents how to make more, where it came from, where it is going
\item
  Technology is not really deployed until you can create it with zero
  federal reserve debt or consumption of mined or extracted material. To
  deploy a technology is simply to make it and have it get used, and you
  must spend zero money to make that happen. Selling it after that is
  optional, and can be done for workers to get central bank debt
  currency but can also not be, and all parts can float in and out of
  different value circles(more on this later in this work)
\item
  Absolute precision will scale linearly with scale, meaning that we
  might keep just 10\% relative precision at different scales, with
  gross motion at 1 meter with a few cm uncertainty, then a few cm
  motion with a few mm precision, on down to 1 nm motion with 1 angstrom
  precision.\\
\item
  Every piece of technology should be as versatile as possible, with
  clear and easy instructions encoded in it for many uses
\item
  We will not build or work with those who build antipersonnel weapons.
  Drones and other machines are fair game as targets, however
\item
  Every technological component should have the maximum possible number
  of uses, and should be cross referenced with other instances of itself
  so that the user can find out those other uses instantly, and this
  should be true of all the sub-components of a technical artifact
\item
  Every technological artifact and component should tell a personal
  story, connected to users, builders, and artists.
\end{enumerate}
