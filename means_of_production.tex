\subsection{Slipping Between the
Cracks}\label{slipping-between-the-cracks}

Where can we build? Where can we work and play and live? Between the
cracks. Between the cracks of the industrial system, between the cracks
of empire, between the cracks of cement and steel. Just as we seek to
build from the discarded materials and energy of the old world, we seek
to thrive and grow in the spaces they have cast aside.

As with materials, those spaces are incredibly rich in every possible
way, when looked at outside the value system of capitalism. Often these
spaces are considered off limits by the powers that be, and are
supposedly private property. However the spaces I'm thinking of are not
monitored and indeed must be ignored for the system to work.

The foul waste filled waterways of suburban New Jersey are one example.
Often spaces under freeway bridges are ignored and abandoned yet
centrally located in a city. Abandoned factories are a fantastic place
to re-start industry in the places abandoned and destroyed by organized
capital. The dark, unkempt corners of various parks have places where
someone can vanish from sight and be centrally located with access to
people and infrastructure.

In general, our first choice for location will be somewhere that has
running water. This could be either a flow of fresh water or a tidal
flow, but it is important both for a source of unlimited energy and for
a source of materials. Deep water is useful for being able to move very
large freight such as salvaged trucks or airplanes found underwater into
position to work on in the production zone.

Water also serves as a source of coolness for protection against hot
climates. Evaporation can be used for added cooling power using water
driven pumps. plumbing which can be taken apart, moved, and put back
together should purify and move the water around, making hot bath water
and god drinking water, as well as building the input for the biological
and chemical reactors which will process human waste.

After water and energy, concealment is probably the next priority for
the ``between the cracks'' model. The easiest way to do this is to be
very low key and work on existing land. If you can camp somewhere, and
the Trash Magic industrial production is quiet and small enough, perhaps
you can simply do it un-noticed wherever you happen to be. Setting up in
the factories the industrial system has abandoned en masse in places
like the industrial Midwest of the USA is another great concealment
strategy.

Camouflage in design is also an important tool. We should be prepared to
build infrastructure that does not look like infrastructure. One of the
ways in which the symbology of Trash Magic can be used is in an
extension of Hobo symbology which allows for cryptic marks to indicate
what infrastructure exists where. Part of the art we study is building
working technology which looks like the environment because it simply
\emph{is} the environment: boulders, logs, sticks, rocks, patches of
seaweed, etc. Substantial infrastructure for industrial production of
all kind can be designed, built, and deployed in this manner where only
the initiates in our Magic will even know it's there.

\subsection{Bring the Means of Production to the
Action}\label{bring-the-means-of-production-to-the-action}

Most communists and anarchists direct us to turn the factory into a
place of political action. I propose to do the opposite: to bring the
means of production to the action. Where there are protests or
occupations or refugee camps or war or poverty, Trash Magic can shine a
light in the darkness.

One of the great tragedies of every radical commune project is when the
forces of Law and Order inevitably come in and destroy everything. In
the case of Occupy, there were libraries, various other services, first
aid tents, all sorts of art and carefully built spaces, which were all
deliberately destroyed by the New York Police Department in their
repression. I see this as very avoidable. Rather than building static
infrastructure which mirrors capitalist infrastructure, I propose that
infrastructure built during various occupations and insurrections should
always be dynamic and mobile. And \emph{all} of it should be art of the
kind which can be easily gifted to others, to spread what has been made.

practical considerations, examples, actually go do it and record it and
put it on youtube

\subsection{Production in Autonomous
zones}\label{production-in-autonomous-zones}

One of our goals is to erase arbitrary lines between things that are
currently separated. Just as some people have tried to erase lines
between protest, occupation and party. I want to erase lines between
industry and art, between protests and factories and workshops and
squats. Anywhere there are people and materials there can be industry.

It's worth mentioning that I don't mean just crafts or hobbies or art in
the current definitions. Part of what separates industry from those
activities today is how they all scale. Art gets its value partly from a
deliberate non-scalability. Crafts are almost deliberately set up to be
non scalable as well, to create some kind of perverse joy in doing
things slowly and with a lot of specialized skills. One speaks of a
``craftsperson'' as someone who has mastered some difficult special
skill, and who therefore has special privileges associated with that
skill.

In Marx's day there was such a thing as an industrial worker, and maybe
in some places there still is. The industrial worker is part of a larger
whole which uses economies of scale to change how people, energy and
materials move in such a way that it will always beat out other forms of
production on efficiently and ``price''. This has led to a historical
dead end as the capitalists have carved up the global working class so
effectively. And good riddance! Do we really all want to work in some
giant factory doing identical boring tasks for many hours, even if the
IWW ``owns'' the factory and we all have free food and health care? Fuck
that future. We bring the factory to the streets where the party is,
inject art and culture in it, and make it able to thrive and grow fast
in the current world.

Here's how it happens. Anywhere there are people, energy and materials,
we just start building industrially and creating art as part of that
process. We build processes and document them(this used to be called
culture) which can be spread and expanded quickly, which allow any group
of people with minimal skills to rapidly build an effectively infinite
inventory of useful industrial products such as air conditioners, water
purifiers, massagers, grinding tools, communications infrastructure,
blenders, coffee machines, electric wheelchairs, soaring surveillance
drones, and medicine. All these goods are immediately entered into the
global decentralized database of free artifacts, which allows them to be
immediately taken by courier by hand to users who absorb it instantly
into society.

This totally changes the balance of power in any occupation. If instead
of occupying the center of town and putting ourselves in conflict with
current society we imagine a bunch of yuppies having to go down into a
Sacrifice Zone to get some awesome artifact they can't get anywhere
else, which they also can't pay fed debt for. There is no transfer of
fed debt or ``ownership'', so all the normal regulations that apply to
commerce do not apply. We slip between the cracks to build up the
factory, make stuff, absorb trash, improve the environment by putting in
infrastructure easter eggs, and disappear. Often the people who come
together to do this will simply not exist as a coherent organization
before or after the industrial/art event.

The powers that be know how to protect ``property'' and to keep the have
nots from getting it from the haves. Much of this has to do with
regulating money. What they do not have experience with is free people
giving away free stuff from trash and ambient energy in on and around
their system. They're prepared for a broken shop window, but not a free
beer fountain in the park. They're prepared for a black bloc in the
middle of the town square but not a boat factory in the middle of a
polluted-to-death river. They are prepared for half a dozen commercial
surveillance drone sent to spy on the cops. They are not prepared for
10,000 soaring drones built from trash, soaring over the dead land of
the American West looking for pollution and mapping it for future use by
our industry.

And this process is within reach now!!! I still think the first
industrial process is the coil winding process which is used to make
more of itself. This means both a coil winding machine and the power
tools needed to quickly break down electrical appliances to get the
copper wire out and the infrastructure required to track down rare earth
magnets, as well as power tools to make lots of Skeletron and plastic
parts quickly. So this means drills and grinders and saws and also heat
tools for working plastic, grinding tools for taking stuff apart, and
good sensors for tracking down magnets. Also free decentralized access
to all the needed data. Energy must be ambient, not oil or human.

This set of tactics then informs the overall strategy and vice versa. It
tells you where to occupy and for how long and with whom, at least to
some extent. We need ambient energy. That means the sun, the wind, and
moving water. Moving water is usually going to be the best choice
because the energy density can be so high. With 1000 times the density
of air, a relatively slow river can be much better than even pretty fast
wind. And way more pleasant to work around. Also waves and tides can be
used, as well as in some cases water that has been pumped uphill over a
long time before the establishment of a industrial occupation.

We reflect the industrial occupation of today through the looking glass:
rather than not building stuff in a factory we build stuff in a
not-factory.

So the first choice for a site is on flowing water, with tides and waves
especially helpful. Also note that natural water, even very polluted
water, is also a source of many useful industrial feedstock. At minimum
you have H2 and O2, but usually a vast wealth of other chemicals. So a
very polluted wetland in the mouth of a river is an ideal site. With a
combination of skeletron and plastic we can build an amphibious set of
shelters and transports and food and water production which add up to
self sufficiency.

Then we need materials, raw materials with a clear path to an
industrially produced artifact or set of artifacts and raw materials to
be moved by courier to another post capitalist industry node.

metal and plastic. And wood. And stones. This can be many places. Rivers
with trash in them, with littoral robots that go out and find it is easy
pickings. Also any dump of car or electronics related junk by a river or
lake or sea. And there are so many of these! Sacrifice Zones are often
near water. And usually have unlimited trash available.

We roll in, we build and distribute, set up infrastructure easter eggs,
and move on. While where there, we create a one-off unique culture for
that time and place, which propagates through the physical artifacts
which carry data that includes the artistic culture of that unique time
and place. This also means that the phenomenon that replaces the current
protest model can be more long lasting. Imagine if any of the famous
protests or occupations, such as for instance the AIM occupation of
Alcatraz had been run this way. You could, today, use an artifact with a
piece of iron from a rebar salvaged from Alcatraz and painted by one of
the occupiers there. Such an artifact could then have been used for an
electromagnet in a big motor that ground coffee beans in Zuccotti park
during Occupy Wall Street, which was then incorporated into a sort of
Jawa art car that roams the toxic waste deserts of Arizona, collecting
minerals for another future project, all with added stories and media
and art.

How different this would be! We could all be participating in various
insurrections, art communes, famous science experiments, and huge
parties at the same time, endlessly remixing artifacts that carry all
that culture with them.

I need to find the Sacrifice Zones that exist in the coastal waters of
the East coast.

Searching my memories of such places in Souther Connecticut and also
looking at maps and charts of coastal DE and MD, I'm reminded that
simply finding the ``free'' material input in such a place is non
trivial. What i think I propose instead is the same courier system used
to distribute artifacts is used to acquire raw materials from the trash
of mainstream society.

Also, if production happens in such coast waterways but materials come
from elsewhere it should be possible to disappear. A combination of
counter-surveillance to always monitor the monitors and camouflage and
totally mobile amphibious infrastructure should make it possible to
avoid detection in un-used land indefinitely. This should be possible
all over the world, anywhere there is a fractal water system. The areas
around Boston, NYC, DC, the SF Bay Area and Seattle are all like this,
as well as many of the great cities around the world.

Trash Pirates. Southeast Asia has marine sacrifice zones where ghost
ships with slave crews fish for the grocery stores of the rich world. If
a guerrilla industrial movement were to appear in this environment with
vastly superior technology to the capitalists, we'll see very rapid
change with no physical opposition from the nation-states. Why? Because
they have built a system where they have a vested interest in these
lawless zones existing. They have to either impose the rule of law on
these places and lose their slave-caught industrial fish slaughter or
they have to accept that our pirates can operate outside their ``laws''
just as readily as our capitalist enemies.

What if Somali pirates could offer legitimately better employment than
the European companies the crews of the hijacked boats work for? it's
hard to negotiate for ``hostages'' who don't want to return, and
dangerous to negotiate for them if when they return they all just quit
and disappear into some swamp. Let's fill in all the spaces the
capitalists have chosen to neglect with new industries that combine art
and culture and science and technology as one thing!

\subsection{Life in the Delta}\label{life-in-the-delta}

The future of humanity is in the deltas. just as the past. And it's
easy. SO many cities have out of the way places an amphibious trash
magic industrial culture can flourish without detection. Freight
transport powered by tidal energy driving electrochemical cells can be
used as a universal industrial supply chain, with vast amounts of trash
gathered for free from underwater salvage and swamp and wetland salvage.
Distribution of goods into the capitalist economy in the heart of a city
via water front parks can then easily happen, also under the radar. By
under the radar, I literally mean under the actual radar, with boats of
such a low profile that they are not distinguishable from wave action by
radar. Fabrication will be right on the edge of water and air, with
object able to be moved in and out, water to be sprayed and pumped and
mixed.

The capitalists have had nanotechnology all wrong. They have been
looking for a clean technology with perfect control. That's wrong. You
want only fractal control, and very dirty, to in fact eliminate the
concept of dirty. Dirty is a capitalist delusion. Must look beyond it.
Under water, fire is also less of a hazard. H2 and O2 plumbing
everywhere, as well as compressed air, fresh water, DC power, various
materials which can be sent in tubes via plastic cells that get pushed
along and tracked. Just the ability to make QR codes in plastic combined
with floating plastic and pumps can make a amazing network demo. Also
for data transmission, when you have material transmission like this,
it's always trivial to send data by putting a piece of physical memory
onto a boat that runs along the channel just like in the pure
information based networks of today. Thus one of the many lines we seek
to erase that arbitrarily divide the world is the line between data and
not data. Data is another capitalist delusion. Information is physical.

Also agriculture. If it were in stormy seas or tidal shallow water with
strong currents it should be much higher density. With both the atomic
feedstock of seawater and the energy content of the tides and waves,
infinite amounts of fresh water, minerals, nutrients, and light(possibly
from electric lights, to get 24/7 underwater agriculture), also things
can be 3d with light generated electrically, water coming in from all
sides, temperature control. The density of crop cultivation should go up
by way more than an order of magnitude, probably at least 2 orders of
magnitude. Thus a few acres of swampy wetlands in a strong tide with a
good river current could sustain hundreds of people comfortably if the
infrastructure is built right. And since it's all mobile and modular and
can be built from trash, even if we all have to move or the State takes
the stuff, infinite infinite.

In addition to deep ocean and river delta areas, this process can build
up land out of the ocean where it is shallow as it often is in the
tropics. Trash can be built up into reefs of industry, designed to draw
energy out of and thrive in storms. Total global game changer.

\subsection{Guerrilla Fairy Art}\label{guerrilla-fairy-art}

I have figured out the nature of the first phase of technology
development: guerrilla faery art. I've been getting distracted by the
long term goals of functionality for industrial production, but for this
first volume aimed at non technical readers, it makes sense to focus on
technology which will make sense and be obviously worth spreading:
guerrilla faery art. What is this? Art outside the capitalist system,
installed without permission, built from trash and powered from freely
available energy, and with a view toward exposing people to the of magic
of the physical world. There will be oscillators and motors and pumps
and strobe lights and magnetic pickups and all kinds of blinking lights
and speakers for sound and microscopic views of living things.

The electrochemical probe and full robotic system belongs to the second
volume on Trash Magic. That is geared to people who want to delve deeply
into the way electromagnetic trash magic works, focusing on fluid ion
transport to interact with living systems, along with the basic
infrastructure needed for a good life. The more advanced stuff will be
just described in the first volume, not built out with detailed plans.

What does this mean for things to build?

Materials and how to mount things in place matter. This gives me an
excuse to go down to all the creeks and find the right sticks and rocks
and trash locally that can be repurposed for an installation. Some
missions will require stealth.

Viewing of microscopic objects must be extremely robust and require no
turning on or off or care on a day to day basis. Obviousness is key
here, the view port has to be so obvious that everyone will
automatically use it. Also the subject has to naturally flow in
constantly, with some trickle from a living stream so that something
interesting, whatever the subject is, is usually present.

What specifically needs to get built to have finished products, and
where do they go? Some things will be deployed in wild areas, some in
urban areas, and some will be gifts to artists.

A tentative and partial list of Guerrilla Faery Art:

USB charger with water wheel water wheel that generates electricity
which drives oscillator stick with rocks on it, just vibrates forever
with feedback same, but with LEDs with a pattern to make 3d POV art in
the water water wheel turns triboelectric generator using bottles and
such to build up high voltage which creates an arc over the water
between aluminum covered plastic bottles, very visible at night! art
piece as gift where a vibrator vibrates water, making waves, which can
be observed using a strobe, and turned into audio with a magnetic float
and amplified magnetic pickup. With the magnifier built into the
wood/plastic/stone water containers, this connects the main technologies
if it's USB powered, and is the perfect Main Gift for this phase. 3d
manipulator with 3d input, hung from a tree or bridge over the water,
which powers all motors and control circuits. Anyone happening by and
seeing the setup can grab the input rock and move it around, which will
drive the moving platform around in 3d space above the water. This probe
can have the crude sonic electrochemical probe tuned to respond to depth
in the water, so that the user can make sound by controlling the probe
around in the water. Here art, science and technology are all one thing,
built from trash, and in a public place with no declared ownership.
water channel with strobe and vibrational drive for visual effects at
night, driven by water wheel, runs all the time evaporative cooling
refrigerator driven by water wheel hotplate driven by water wheel warm
water pool heated by water wheel and generator steam powered organ using
tubes and steam generated from water wheel datalog of creek which can
connect to phones and twitter

Focusing on the main thing for now it's probably the USB driven art
piece without the generator, just a wall charge for a off the shelf lipo
battery, or left plugged in. A wave tank with a strobe can have a
tunable 2d shape projected by the sun down onto an area, with musical
output based on the wave patterns. This could be installed in a tree,
projecting through glass, with water piped from the top of a waterfall.
But what powers it? No, I need the charger for the guerrilla
installation, but not for the art gift.

Art gift should be simpler than that, project up and along the side,
with lights under transluscent plastic in the stick. Vibrator stick with
rocks on it bounces, with a stick that can be adjusted to agitate the
water with different wave shapes and frequencies and amplitudes. The
magnets and rocks can also be moved to change the properties. Water
propagates down carved channels in a fat bottom stick with the drive
stick bolted to it as well as the bouncing stick which is fixed at the
end opposite the water. Lenses can be put above the water to magnify
what is in it as well as to project light in various directions both for
art and for observation. A little wave pool at the opposite end of the
water agitator has a float with a tiny magnet in it, and the audio flux
amplifier is wound around this pool, so that the sound is picked up and
amplified and has an audio out socket. A beautiful carved wooden knob is
used to adjust the strobe properties by changing a 555 circuit.

This is the first thing! Build this art gift first, before the water
wheel, it's self contained and can be distributed and used in classes I
can teach and spread the work. Lack of water wheel is not serious for
most people since they charge devices anyway with USB and can get a lipo
at a gas station for 10 dollars.
