To the beat of the drum:

\emph{ROBOTS that turn junk cars into robots}

\emph{that turn junk cars into ROBOTS}

\emph{that turn junk cars into ROBOTS}

\emph{that turn junk cars into ROBOTS}

\emph{that turn junk cars into ROBOTS}

\emph{that turn junk cars into ROBOTS}

\subsection{Cars are Death, Death to
Cars}\label{cars-are-death-death-to-cars}

Cars are the enemy of humanity. Every year in the US cars kill over
40,000 people, and maim countless more, similar to the \emph{total}
carnage for the U.S. of the entire Vietnam war which lasted many years.
Globally, the death toll is well over 1 million, or 10 million in a
decade. 10 million dead.

And that is just the beginning. Cars are central to the industrial
system which has crushed our humanity. A huge amount of our oil based
economy is used by the car system, adding to climate change massively,
as well as bad urban air which kills millions worldwide. Cars create a
society in which anyone who cannot drive is disenfranchised, punishing
anyone without perfect health and significant funds as well as the
willingness to actively destroy the world and possibly kill living
things just to get through their day. Cars have filled up the USA with
enough pavement to provide solar power to the entire nation(no small
feat given our absurd energy consumption now). Runoff of oil and other
toxic chemicals leaks from cars into every water system in the world,
poisoning every possible ecosystem.

The companies that produce cars are some of the most evil on the planet.
Several of the major global brands, including all the German ones, have
actively participated in genocide, an act for which they have never been
properly brought to justice. The endless stream of minerals required to
feed the input end of the planned obsolescence conveyor belt also
destroys the world in the ways that mining always does, with its usual
disproportional impact on indigenous people around the world and on many
other marginalized populations.

In the U.S. and many other countries, car companies actively work to
undermine democracy and civil society, campaigning to make sure society
is built around the profits of their companies rather than basic
principles of free movement of people. The ability to move from one
place to another within a city for free should be a basic part of any
social contract that people would actually consent to. Car companies
have built a society where there is no universal social contract in
regards to mobility: all mobility is held hostage, under threat of
violence, by a group of psychopaths(all car makers) who force all
transport to make them money. Even ``public'' transit is always based on
giant machines made by the same monsters, and is deliberately priced
high enough to make sure the poor pay at least as much per mile for
getting around as those who have the money to buy into the car system.
Every time there is an economic downturn, the corporate backed local
government will use that as an excuse to further crush the lives of the
poor, raising fares and cutting services at the very time those without
resources are likely to be the most desperate. Again, this shows the
fact that there simply is no social contract in the modern industrial
city which all citizens consented to. There is only the raw law of
force: whoever has the most control of the industrial machines has the
power of life and death over everyone else. Of course the rulers dress
this up in nice language about the ``rule of law'', but there simply is
no such thing. It's a costume raw force wears in our world.

So the car companies and their collaborators in government are enemies,
and cars form an almost living enemy of humanity world wide. What should
we do about this? The Trash Magic answer is always the same: first find
the trash stream(which always exists under capitalism since destruction
is inherent in their ways) and then find ways to organically incorporate
this into something good rather than bad.

\subsection{Magical Answers!}\label{magical-answers}

And what treasure there is in cars! Name any precious metal or special
type of polymer or gas fitting or mechanical device and you can find
them in a car. A single automobile also has numerous computers of all
kinds, which can be stripped and used for integration in our electronic
systems. And given the spectacular waste of the current system, cars
really are free: while there is a used market for junk cars, it's clear
that for society as a whole the global stream of junk cars, like other
industrial waste streams, is a net liability not asset. This negative
value creates a global and ongoing opportunity for us to get the parts
we need from it.

Another advantage of using car parts for industry is the way in which
the car is standardized. There exist millions of units all over the
world of certain popular car and truck models, and it can be possible to
very accurately duplicate a complex design which uses parts from a
certain make and model because of this.

In particular, the parts of cars are great for building robots. As
discussed earlier in this work, robots in the right hands can have a
fantastic positive impact on the human condition. And this is really
what this story is about: the robots built from cars which destroy cars.
If they are easy to build, and it's easy to \emph{teach} people to build
them, they can self reproduce, creating an exponential destruction
vector through all cars globally.

Robots must be designed and built and grown which first find cars, then
rip them apart for scrap, then sort and catalog the parts, then re-form
them into more robots. These can all be different kinds of robots,
possibly used separately, possibly together, made by many people with
many methods. The point is they should be easy, and create positive
value(unlike what the capitalists build, ever.)

\subsection{End Game: End All Cars}\label{end-game-end-all-cars}

Destroy every single car. Rip out the individual atoms. Rip them apart.
Smash the engines, destroy any vestige to show that they ever existed.
Rip up the roads. Build structures to live and work and grow crops in.
Make them all green, smash them but don't replace them with private
property, this is a wedge to build more non private property space.

The end game is this: when the thing a car turns into after it ceases to
be a car has greater value than even a functioning car, cars will start
being consumed by our technical ecosystem even before the end of their
capitalist lives. As non-capitalist ways of living expand, the companies
that make cars will be increasingly starved of the consumers they need
to keep building and growing their death machines. Eventually the
companies will die and the existing cars will be destroyed faster than
they can be made, eventually making the capitalist industrial system
simply physically unable to keep making any cars. Without enough car
consumption to fund the corporate government, there will no longer be
military force protecting roads and they can also all be ripped up for
use by humans and other living things.

\subsection{Free Lives Don't Need Cars}\label{free-lives-dont-need-cars}

At this point, the capitalist will whine: ``but without roads and cars
how will I get around?'' First of all, you have to ask why you need to
drive around all day in the current system. Do you really need all that?
Do you need to go to an office miles away to do things no one needs
done?

No. No you don't, and that's how all your errands are. Errands are
unpaid labor you have to do for industrial civilization. Stop doing
errands. Stop working. Build and grow what you need, then go have
adventures as needed. You should have everything you need for a good
life, except adventure, within easy paddling distance of your main
bathtub/bed. You travel when you want to for adventure or when you are
part of a larger migration which will use larger vehicles, mostly boats
and giant spider pods, for moving whole populations.
