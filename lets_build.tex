\subsection{let's build this!}\label{lets-build-this}

this chapter needs a ``product'' focus.

It will be these three things.

\begin{enumerate}
\def\labelenumi{\arabic{enumi}.}
\tightlist
\item
  The vibrational drive with strobe driven by USB battery, pendulum and
  spring toys
\item
  Noise stick driven by USB battery
\item
  USB charger guerrilla art whirligig, with LEDs and glowing rocks
\end{enumerate}

build a thing, start the story, teach it on, and deploy, for the three
things, with examples and pictures.

I will list the things I'll leave for other projects/publications:

\begin{enumerate}
\def\labelenumi{\arabic{enumi}.}
\tightlist
\item
  robots that roll around
\item
  3d input manipulator
\item
  3d probe
\item
  stepper motor
\item
  water powered drill
\item
  microscopy
\item
  fluidic pumps
\item
  cable to move goods around mechanically driven by whirligig
\item
  water pump driven by water
\item
  generic air pump for inflating anything, which is also generic water
  pump
\item
  electrostatic generator driven by whirligig
\item
  stone cutting machine
\item
  boat
\item
  heated composting reactor
\item
  oxygen and hydrogen generator
\item
  electrochemical probe
\item
  optical free space communications system
\item
  soaring high voltage drones
\item
  temperature regulator using stepper motor and solar concentrator with
  parabolic mirror from trash
\item
  thermometers
\end{enumerate}

\begin{itemize}
\item
  how to make a thing
\item
  how to ship a thing
\item
  how you can get finished goods and join a value circle
\item
  how you can make stuff and start a value circle
\item
  how you can do a research project and start a value circle
\item
  how to transport a thing from the maker to the user
\item
  how to find your ambient energy resources
\item
  what you can do to spread the word and build community
\item
  Other academic work you can contribute as a scholar
\item
  Machines you could build outside this system which can start a new
  value circle
\item
\end{itemize}

This section should have some basic assembly plans and information to
get started even without the second volume.

Things to make here:

\begin{itemize}
\tightlist
\item
  vibrational oscillator with musical pickup
\item
  3d input device
\item
  3d manipulator, linked to input device
\item
  wood cutting for circuits techniques
\item
  plastic welding techniques
\item
  waterwheel generator to 5V USB charger
\item
  skeletron guide, with pictures, cartoons, detailed instructions,
  several plans:

  \begin{itemize}
  \tightlist
  \item
    tent
  \item
    water wheel
  \item
    tripod for manipulator and probe
  \item
    vibrational musical instrument
  \item
    boat
  \end{itemize}
\item
  high voltage generator
\item
  strobe microscope
\item
  LED art vibrator display
\item
  basic electrochemical probe
\item
  temperature regulator with hoist and thermometer and fire
\item
  ambient art pieces powered off water and wind that use electricity to
  do things which can be repaired forever by anyone and moved and
  rebuilt and replaced by anyone
\item
  vibrator with polished stones as massager
\item
  Josephson junction pendulum
\item
  build a heated shit reactor with a giant tube and air pump to send
  gasses to the top of a tree
\end{itemize}

\subsubsection{circuits:}\label{circuits}

\begin{itemize}
\tightlist
\item
  generic vibrational feedback drive
\item
  magnetic pickup to audio output for music
\item
  stepper motor driven by potentiometer so knob goes to wheel directly
  with one Arduino per motor+knob combo--three of these and two tripods
  is the standard manipulation space for a generic tool
\item
  strobe with trigger and 555 delay and knob to tune delay, can be used
  for microscopy with vibrational drive of water position, control of
  phase delay can control focus electronically with no mechanical focus
  knob needed
\item
  POV on vibrational object to make 2d image
\item
  power supply circuit that turns output of AC generator coil into 5V
  regulated power
\item
  step up board that charges a 50 V capacitor from AC generator coil
  input
\item
  stepper driven by two digital inputs and two buttons for one direction
  and the other, with feedback coils for free running operation at
  variable torque
\item
  electrochemical probe with audio output for conductance based on an RC
  oscillator and LED array to show average voltage, two joysticks to
  control the average voltage and the amplitude of the oscillation(with
  feedback): one micro controller, one neopixel array, a speaker with
  transistor drive, two joysticks, various connections with different
  capacitors and different wires and possible probes
\item
  digital thermometer with serial bluetooth readout and up/down digital
  motor control output lines and LED neopixel indicator for direct
  temperature readout, 5V Usb drive
\item
  circuit to drive 120 VAC power plugs with output from a dynamo, with
  decent sized water wheels, I think this a UPS with a 12 V supply for
  the 12 V battery, should be made from trash
\item
  interface between smart phone and pair of stepper motors for generic
  two motor robot control
\end{itemize}

this is a complete enough technical set that it can have a huge physical
impact on the world even without the next volume, and then the next
volume is truly a work for people who want to push the boundaries of
science and trash magic together, with emphasis on applied physics.
Maybe there will be a volume three that is biological and chemical

encourage collaboration so that those who don't want to learn
electronics can get free electronics from those who do, and can find
other ways to contribute, with art, assembly, craft, and the important
courier service that needs to happen for distribution.

Recruiting couriers might be the most important of these construction
tasks, as I will be always building more artifacts myself and spreading
the teachings of how to do it face to face, which can't spread as fast
as the book can. The book can spread \emph{very} fast, and build a
courier network so that physical artifacts can also if the book spreads
the courier network very fast(it should also explicitly encourage
readers to spread the word fast), and I can make artifacts fast, the
limiting factor in growth will be the spread of the craft of artifact
creating and the use of the new things.

Thus my problem is the same as that of Kano in forming the Kodokan: how
I teach the first students and how I organize the information conveyed
to them shapes how they pass it on, and that will propagate through the
whole subsequent series of events. How did Kano start? How did Ueshiba?
Why was Kano able to build a more functional system than Ueshiba?

\subsection{public guerilla art}\label{public-guerilla-art}

how to do installations, how to fix them how to document them, how to
expand them how to move them, documentation of installations around the
world, USB chargers are best to start with

from the blog:

I have figured out the nature of the first phase of technology
development: guerilla faery art. I've been getting distracted by the
long term goals of functionality for industrial production, but for this
first volume aimed at non technical readers, it makes sense to focus on
technology which will make sense and be obviously worth spreading:
guerilla faery art. What is this? Art outside the capitalist system,
installed without permission, built from trash and powered from freely
available energy, and with a view toward exposing people to the of magic
of the physical world. There will be oscillators and motors and pumps
and strobe lights and magnetic pickups and all kinds of blinking lights
and speakers for sound and microscopic views of living things.

The electrochemical probe and full robotic system belongs to the second
volume on Trash Magic. That is geared to people who want to delve deeply
into the way electromagnetic trash magic works, focusing on fluid ion
transport to interact with living systems, along with the basic
infrastructure needed for a good life. The more advanced stuff will be
just described in the first volume, not built out with detailed plans.

What does this mean for things to build?

Materials and how to mount things in place matter. This gives me an
excuse to go down to all the creeks and find the right sticks and rocks
and trash locally that can be repurposed for an installation. Some
missions will require stealth.

Viewing of microscopic objects must be extremely robust and require no
turning on or off or care on a day to day basis. Obviousness is key
here, the view port has to be so obvious that everyone will
automatically use it. Also the subject has to naturally flow in
constantly, with some trickle from a living stream so that something
interesting, whatever the subject is, is usually present.

What specifically needs to get built to have finished products, and
where do they go? Some things will be deployed in wild areas, some in
urban areas, and some will be gifts to artists.

A tentative and partial list of Guerilla Faery Art:

\begin{itemize}
\tightlist
\item
  USB charger with water wheel
\item
  water wheel that generates electricity which drives oscillator stick
  with rocks on it, just vibrates forever with feedback
\item
  same, but with LEDs with a pattern to make 3d POV art in the water
\item
  water wheel turns triboelectric generator using bottles and such to
  build up high voltage which creates an arc over the water between
  aluminum covered plastic bottles, very visible at night!
\item
  art piece as gift where a vibrator vibrates water, making waves, which
  can be observed using a strobe, and turned into audio with a magnetic
  float and amplified magnetic pickup. With the magnifier built into the
  wood/plastic/stone water containers, this connects the main
  technologies if it's USB powered, and is the perfect Main Gift for
  this phase.
\item
  3d manipulator with 3d input, hung from a tree or bridge over the
  water, which powers all motors and control circuits. Anyone happening
  by and seeing the setup can grab the input rock and move it around,
  which will drive the moving platform around in 3d space above the
  water. This probe can have the crude sonic electrochemical probe tuned
  to respond to depth in the water, so that the user can make sound by
  controlling the probe around in the water. Here art, science and
  technology are all one thing, built from trash, and in a public place
  with no declared ownership.
\item
  water channel with strobe and vibrational drive for visual effects at
  night, driven by water wheel, runs all the time
\item
  evaporative cooling refrigerator driven by water wheel
\item
  hotplate driven by water wheel
\item
  warm water pool heated by water wheel and generator
\item
  steam powered organ using tubes and steam generated from water wheel
\item
  datalog of creek which can connect to phones and twitter
\end{itemize}

Focusing on the main thing for now it's probably the USB driven art
piece without the generator, just a wall charge for a off the shelf lipo
battery, or left plugged in. A wave tank with a strobe can have a
tunable 2d shape projected by the sun down onto an area, with musical
output based on the wave patterns. This could be installed in a tree,
projecting through glass, with water piped from the top of a waterfall.
But what powers it? No, I need the charger for the guerilla
installation, but not for the art gift.

Art gift should be simpler than that, project up and along the side,
with lights under transluscent plastic in the stick. Vibrator stick with
rocks on it bounces, with a stick that can be adjusted to agitate the
water with different wave shapes and frequencies and amplitudes. The
magnets and rocks can also be moved to change the properties. Water
propagates down carved channels in a fat bottom stick with the drive
stick bolted to it as well as the bouncing stick which is fixed at the
end opposite the water. Lenses can be put above the water to magnify
what is in it as well as to project light in various directions both for
art and for observation. A little wave pool at the opposite end of the
water agitator has a float with a tiny magnet in it, and the audio flux
amplifier is wound around this pool, so that the sound is picked up and
amplified and has an audio out socket. A beautiful carved wooden knob is
used to adjust the strobe properties by changing a 555 circuit.

This is the first thing! Build this art gift first, before the water
wheel, it's self contained and can be distributed and used in classes I
can teach and spread the work. Lack of water wheel is not serious for
most people since they charge devices anyway with USB and can get a lipo
at a gas station for 10 dollars.
