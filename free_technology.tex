\subsection{What Does it Mean for Technology to be
Free?}\label{what-does-it-mean-for-technology-to-be-free}

Free means that a thing can be created with only labor and the waste
products of the old world or renewable products of the natural world,
using information that is available to the public both physically and
logistically.

I will start with a list of what makes technology non-free. Since this
is a manifesto, it makes sense to call out what the problems are that I
aim to work on with this project.

What does it mean for hardware to be non-free?

\begin{itemize}
\item
  If someone claims the legal right to control who can make a thing it
  is not free.
\item
  If materials mined or otherwise extracted from the Earth are needed to
  make a thing it is not free
\item
  If professional expertise that cannot be learned in a short time from
  clear online instructions are required to make a thing it is not free
\item
  If a tool from the consumer capitalist economy is required to make a
  thing(e.g.~a 3d printer from a factory) it is not free
\item
  If the fabrication of a thing requires the use of energy from the Grid
  or non renewable sources, it is not free
\item
  If a thing cannot be re integrated into the industrial ecosystem in a
  modular way after its lifetime it is not free
\end{itemize}

What about free technology, what is that?

\begin{itemize}
\item
  A free thing can be made from readily available waste \emph{streams}
  of the existing industrial capitalist system
\item
  A free thing is not patented and is disclosed publicly in sufficient
  detail to make patenting it illegal
\item
  A free thing has publicly shared non copyrighted instructions which
  enable a non expert to learn what they need to learn to complete the
  construction of the thing
\item
  A free thing can be fabricated in a scalable way, from single units up
  through millions of units, with automation at large volume using
  robots built from same technology
\item
  A free thing uses only ambient energy to function and to be produced
\item
  A free thing has a post life trajectory built into the design, where
  all components are easily salvaged into other Free Things
\item
  The construction of a free thing must create value from ``nothing'',
  which can then create value outside the world of numerical currency
\item
  An individual thing by itself is free if it is also part of a larger
  group of technologies, which I call a ``complete technological set'',
  which can be used to reproduce themselves and to provide all basic
  human needs
\item
  Free technology does not distinguish between technology and art: it is
  always both.
\item
  Free technology naturally reproduces with the help of people and/or
  other animals. If left out somewhere, people will naturally choose to
  use the thing and information contained in it to make more and to
  continue the development of that technological path.
\end{itemize}

\subsection{What about Open Source?}\label{what-about-open-source}

What is the connection between free technology and ``open source
hardware''? Open source hardware does not at all have to be free: it can
require a vastly expensive factory to actually produce, as long as the
design is publicly available. This maintains the power relationships of
industrial capitalism: the means of production remain safely in the
hands of the capitalists, we are just re-arranging how we share amongst
ourselves. The difference between free and open can be more subtle for
software where it's always free in the sense that it can be copied an
infinite number of times for no cost in principle. Hardware on the other
hand is not just information. Without supply chains that are wrested
from the control of the masters of the system, what is or is not free is
affected very little by ``open source'' hardware.

Another important shortcoming in the open source model is the lack of
demand for the project to be accessible to those outside the technical
guild that built it. This is not as bad as it used to be, but it's still
common practice for ``open'' to mean a thing has horrible documentation
and usability as contrasted to ``closed'' commercial software. What this
really does is \emph{further} enforce the class divisions in capitalist
society by making a hierarchy of who gets free stuff and who doesn't.
Those who are in the software tech guild can get free things that are
unusable to a normal person, and which have such opaque help that no one
outside the guild can be reasonably expected to figure it out.

Avoiding this shortcoming of open source software in the free hardware
project will be a challenge in some cases. This means that if you want
to use something involving the physics of magnets to build a thing, the
quality of applied physics education you make available to your user
determines the freeness or non freeness of your technology. That means
that any free electromechanical technology is not really deployed until
a whole curriculum is made freely available on classical mechanics and
electrodynamics. That curriculum must be held to much higher standards
than are presently applied for college or high school physics education.
It must be very applied, with direct numerical examples throughout which
can be easily run by a novice on any computer or phone. Also it must be
able to cater to a very diverse range of learning styles: hands on,
mathematical, theoretical, visual, etc etc. \emph{All} of these must be
made freely available in multiple open free formats. It must be possible
to do this with printed pages and no computer or with any type of
computer or personal device and no printer(either). When the thing is
built, it must have information printed on it or embedded in some
obvious way, which links back to the main free storehouse of
documentation. That documentation must also be decentralized to prevent
any authority from destroying the information.

This imperative really affects the way that progress moves along. A
working wire coil is not enough. It must be well characterized and
documented with a series of easily accessible physics experiments. There
must be both video and written content showing how to put it together.
These experiments lead to a very fractal level of digression, but in the
end they lead to absurdly robust technology which can be recreated from
scratch by anyone anywhere quickly.

\subsection{Free Everything!}\label{free-everything}

What is free energy? Usually this term is used by various conspiracy
nuts to describe ways of ``getting energy for free'' from something like
the zero point quantum energy or the Earth's magnetic field. Both of
these are nonsense, as are all the free energy schemes presented
throughout youtube and the rest of the Internet.

No, we are told, energy is not ``free''. It has to COME from somewhere.
But this notion is based on a capitalist world view. Energy is deemed
``free'' if you don't have to get it from a mine and labor. Most modern
renewable energy is not free: much labor is expended to build the
infrastructure out of mined minerals which have a finite lifetime and
eventually go to landfill to be replaced by more mining and labor.

But if free energy is energy that can be useful but is not derived from
mining and labor, then free energy can and does exist. Energy not spent
on air conditioning when you build under a shade tree is free energy.
Energy from the sun that warms through your front window is free energy.
And the electrical energy stored in salvaged rebuildable capacitors from
salvaged rebuildable robots storing ambient energy is free.

Capitalist logic always looks for ways to show that things are not
really free, because capitalism is based on the ideas that value comes
from labor and mined minerals. If we approach industrial development
from an anarchist perspective, however, we seek to build technology
which is truly free, where no mineral extraction is implied in its
construction.

A technology is free when it gives more than it takes. For instance a
robot might require a few hours of service from human labor once a year.
But if it does the equivalent of even just a few hundred hours of human
labor it has a net negative cost in labor-value. In terms of minerals if
it is built from minerals that were polluting the world around us, the
mineral cost is negative: as opposed to subtracting value from the land
as mining does it adds value to the land. And finally the energy of the
technology must be free in the sense that it absorbs from something
unwanted elsewhere.

Ultimately what is being built here is a form of artificial life. Life
takes only what can be given from somewhere else. Our technology exists
in a world where humanity is God. This all goes back to the notion that
the structure of our technology is based on the monotheism of its
initial architects. We have built a technological world where Man is God
and only God is above Man(to use biblical sounding gibberish).

But this technology will be alive, will exist as animals and plants do,
without a singular separate God. This means that while it needs humanity
to help it survive at all stages and can easily be controlled by
humanity it will exist on its own and can function to a large extent on
its own, following it's hardware-progammed logic to find what it needs
in the environment to keep living and carrying out its mission.

Free technology is owned by no one. Not only is there no intellectual
property, there is no physical property, except for the Trash Wizard
stick, which might effectively be a part of a Trash Wizards person. The
act of creation of an anarchist artifact is a gift to society of that
artifact. A trash wizard might grab any technology lying around and re
purpose it at any time. Anarchist technology does not recognize the
concept of assigning value to things numerically in any way. Anarchist
technology may get involved in various value circles, having various
types of abstract relationships with various value circles, as codified
in the Data Feed. Anarchist technology is also energy free in the sense
that it always uses ambient energy, be it a set of pedals, a hand crank,
a wind turbine, a steam turbine, a tidal generator, a lightning
accumulator, or a solar concentrator. Anarchist technology is designed
to be as modular as possible, being as friendly with other unrelated
technology as possible. Anarchist technology does not distinguish
between information, energy, and materials--all three are processed as
equal participants in the various flow through the system. Technology is
not to be considered free unless it can be constructed by a small band
of trash wizards using their trash wizard sticks using common source
materials from the waste stream of the old extractionist economy. The
ideology of trash wizardry is that capitalist industry sacrificed itself
for the bounty of our new free world. Mining is dangerous and
destructive and suicidal, but it's done, and we thank our ancestors,
thank their sacrifice and their hard work and the creation of so much
material wealth so evenly distributed(you can find a mineral from
anywhere pretty much everywhere thanks to the spread of capitalist
industrial technology). We give thanks for this great gift from our
ancestors and build a society based on free living on the bones of the
old world. We accept that things will never go back to how they were
before industrial capitalism but that we can live better because of our
mineral inheritance. We accept that the ways of the old world were a
suicide pact, but also that even in a more free world, we can never be
free from change and uncertainty. Ways of life, empires, whole worlds,
climates, continents, will rise and fall, and we cannot stop that level
of cataclysmic change from happening. But we can build an adaptable and
sustainable future based on free values that moves forward into a future
actually worth seeing. We can bring adventure back into the human
condition, as well as acceptance of a huge and uncertain world, and our
role as passengers on it.

Anarchist technology also breaks barriers between customer, worker,
engineer. We eliminate these hierarchical notions. We are people. We
build things as needed and help each other as needed. We tell stories to
express our values with the help of our Data Feed. We break the very
idea of an economy open and build a new way of relating to each other
and existing.

\subsection{A Technological Complete
Set}\label{a-technological-complete-set}

following blog post needs to be cut up and turned into the complete
technical set, with another list and maybe a cartoon

\subsubsection{Destroying the Economy}\label{destroying-the-economy}

That is the goal. It is as it always has been an evil system to force
all of humanity to help evil people to do evil things. End trade. End
money. And end all private property, now and forever.

Fundamentally, as every capitalist will explain, the economy is about
making it easier for people to trade different kinds of things. And it
is of course assumed that you need things from someone you don't know
who wants to trade money for stuff you ``need''(even if that need is
artificial, based on those people controlling all the communications
technology on the planet).

So the way to destroy that is with technological Complete Sets. A
technological Complete Set is a set of technological methods and tools
which allows the users to live without an economy. That means they
already have everything they need with that core technology plus some
work that is not too arduous for them to do(less arduous than engaging
in the outside economy).

A complete technological set has the following needs met:

\begin{itemize}
\tightlist
\item
  food
\item
  clean water
\item
  disposal of human waste
\item
  temperature control inside sheltered areas: heat and cooling of air in
  indoor environment of some kind, construction of those shelters such
  that this needs minimal\\
\item
  energy(use natural heat and coolness from the environment)
\item
  communication/networking/controls/automation/audio/video/VR/AR these
  are the real reasons we need ``computers''
\item
  medicine and drugs
\item
  make any of the tools needed for the rest of this, and do what
  industry might be needed to adapt to changing conditions: more people,
  fewer people, new
\end{itemize}

That's enough. The rest comes from that. And this is very hard and
encompasses a lot of things.

Food is the one people always gravitate towards first, but I think
that's a mistake. Growing your own food does not give independence,
especially if that food is tied to land that is part of the ownership
system. To be truly free you have to be able to get food fast anywhere
with gathering, hunting, and \emph{rapid} and \emph{dense} agriculture.
My guess is that a new agricultural technology will be needed that
integrates the rest of the complete set with food \emph{and} drug
production, since it will all be part of the fractal reactor system,
moving nutrients around as needed to grow both food and also other
things that can be grown like drugs and even carbon nano structures. So
when I put food on here, I'm not thinking of farms I'm thinking of a
huge range of options. For societies that have chosen to live in water,
I'm imagining 24/7 aquaculture driven by high intensity grow lights made
from organic LEDs which are driven by tidal energy, combined with
reactors that get needed nutrients from the sea while removing undated
salt. For deep sea dwellers, the main energy source will be violent wave
action and wind, which can power floating worlds of aquaculture in the
same way.

I propose that the problems that need to be solved for food independence
will be solved as a side effect if we focus on medicine first. This is
one of the ways the capitalists use of controlling us. And they know it.
``Sure'', the capitalists say, ``go live in your hippie tree commune.
But when you need an MRI and some antibiotics or AIDS drugs, you'll have
to come to us and if you don't have federal reserve debt currency to pay
for it we'll let you die.''

As applied physicists it is our job to build the tools that let people
practice medicine. That means chemical testing and processing, growing
of all types of microbe and plant needed for medicine in house with
short lead times, non-invasive imaging, surgery, prosthetics, and a lot
of other measurement tools, as well as the ability to quickly and
accurately access the sum total of human medical knowledge. The last
part will require a complete reorganization of how medical knowledge
works, and elimination of the arbitrary lines between doctor, nurse,
pharmacist, patient, technician, and all the rest. That is a hard
problem, but it has to be solved to destroy capitalism, because we need
medicine to live good lives and the capitalists have one of the most
vile monopolies on that.

So we need a chemical reactor that can work with microbes as well as
chemicals, but this also covers a lot of other useful things! It's how
we get clean water and turn human waste into useful products, including
food, covering several of the points above. It's also how a lot of
manufacturing will happen, because a closed environment of tubes and
chambers and pumps is such a good place for assembler robots to
function.

And what about cooling? We need refrigeration for a lot of things,
including food and medical storage, as well as cooling to make spaces
not too hot to live in. That means pumps, and fluids. If you can pump
and move fluids around you can cool, with any of various working fluids,
including water and some readily available other chemicals like ammonia.
Making ammonia from urine and then using compressors to make coolers out
of that seems like a good choice for a universal basic cooling unit.

Heat should really be the clever use of solar(as in heat, not some photo
voltaics, which I oppose in their current form) as much as possible. And
cooling of human habitat should be the clever use of cool deep water and
cool deep earth as much as possible. The heat is there and the coolness
is there, we just need to think the heat flows through a bit more. And
with private property fetishism eliminated, and the States finally
smashed, migration can be a huge part of this. It is a simple fact of
life that some places are much nicer one time of year than another. One
of the great crimes of the nation-state is forcing humanity to pretend
this isn't true. Migration to a different climate on the time scale of a
season is not hard technologically, it's all politics that stops it. No
borders! No nations! No property!

So now the list above needs to get re-arranged into a list of things to
actually build. Pumps, motors, generators, energy storage electrolytic
cells, energy storage in pumped water, construction of all sizes of
tubes, all this forms the matrix the rest is built in. And I need the
generic assembler/editor technology mentioned before, where manipulators
can cut and weld from the nanoscale up through the meter scale the found
objects thrown away by capitalist society.

That should form the seed. If it's easy to do a chemistry process, build
a good environment for a biological process, and reverse engineer and
edit arbitrary semiconductor circuits, people with expertise on these
things will be able to quickly replicate the capitalist technology they
use now. Most ``professionals'' are being hurt by capitalism now, and
using bad tools that make it hard to do their jobs. Given the
alternative of free and also better technology they'll move over in
droves and drive this thing really fast, we just need to light the
spark, make that first set of tools, and lay down the design rules that
make this progress work well while continuing to avoid capitalism. Part
of how this needs to work is we need tools that people can adopt
\emph{quickly}. A trained doctor should be able to use our medical tools
immediately because their function is obvious, simple, and easy to
modify as needed by a person competent in their trade but with zero
background in our specific technology. We seek to remove the technician
and engineer completely from the process of technology usage.

How does this all add up to destroying the economy? The best people will
jump ship the instant they see that we have a better offer than the
capitalists. The capitalists rely on the exploitation of the
professional class(with lots of perks thrown in to differentiate them
from the working poor) for their system to work. Given a choice, if
people switch instantly to our methods, their system of fear will
crumble. They will keep paying people to do work, but the wages will
have to spiral upwards as the best people refuse to work for money.
Eventually the working class can actually bankrupt the capitalists by
removing their labor from the money system. If the last capitalist wants
to pay the last professional a trillion dollars a year to sell
themselves stuff, so be it. Without the labor of the masses, they're
just another LARP club, and harmless. And that's how you kill the
``economy''.
