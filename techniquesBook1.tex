This might want to go before the one about building specific stuff. Need
the skills first, and this will force me to make it more coherent and
readable and short. It should not read like an appendix, but like an
early to follow story in which you participate by building some things.

\subsection{Finding the right Sticks}\label{finding-the-right-sticks}

Don't hurt the trees! We want sticks that are no longer part of a living
tree but which have not yet been consumed by fungi and other organisms
which turn logs into dirt. Drift wood is also often too far consumed to
be of use, although this really depends on the drift wood. What we want
are freshly fallen sticks from living trees, mostly. And we're looking
for them to be between an inch and 2 inches in diameter, mostly
straight, with not too large knots if possible.

A lot of electronics projects are perfect for sticks about 1 inch around
and 4-12 inches long, so gathering and preparing these is a good idea.
Load bearing parts for larger constructions should be more like 1.5 to 2
inches in diameter, or bigger in some cases.

\subsection{processing sticks into skeletron
etc.}\label{processing-sticks-into-skeletron-etc.}

After locating the sticks, you'll want to saw the ends off flat so that
they're not jagged. I often find that it's easiest to gather sticks by
hand without carrying a saw with you when you go out.

I also find that for this stage it's good to have two hand saws: one is
a big rip saw with huge teeth that very quickly will cut through wood
but is specifically not intended for metal. The other is a much smaller
screwdriver-like saw with a more hacksaw like blade for easy carry and
use on random materials including plastic and metal. This tool is also
useful for removing knots and branches from your main stick branch.

Once you have your sticks of about the right size, you want to shave off
all the bark. This is done with any of various types of pocket knife,
and I find it useful to have a multi-tool of the kind that is also a
pliers and screwdriver and such for this. Ideally you'll do this where
the massive pile of bark and shavings will be useful for something, like
grinding into sawdust which can be used for a compost reactor. At the
very least somewhere there is already mulch will mean you don't have to
clean it up because it's adding to the existing mulch.

\subsection{Finding the Right Plastic}\label{finding-the-right-plastic}

I have found that the best plastic for our purposes is LDPE and HDPE
which stand for low density and high density polyethylene. They are
indicated by the recycle symbols 4 or 2, and are mostly cross
compatible.

The easiest source of HDPE for most of us is bottle caps. Standard
plastic soda bottles which are made of PET or similar plastics that I
find more annoying to work by hand usually have caps made of a opaque
material which is typically some color like red or blue. Anywhere
plastic trash can be found, you can probably find these caps. You don't
care how much they have been smashed, but you do care a bit how dirty
they are. You can always grind or cut off the really gnarly dirt with a
knife or file or similar sharp tool if there is too much crust on the
cap. It's generally a good idea to have a small bin filled with these
caps near your work area.

The other main source of plastic I use for small electronics work
especially is the translucent(but not transparent) plastic generally
used for plastic milk bottles. It is also used for various citrous
juices such as orange and pineapple, so if you don't drink milk that's
probably a better bet. Some 1 gallon water containers from generic
brands of bottled water also come this way, and those can be found in
plastic trash piles by various creeks sometimes. I avoid milk bottles I
find that way due to what happens to milk when it's been out a few days.
I drink milk at home and when the bottles are done I try to immediately
wash them out, rip them up, and put them in my plastic material bin.

\subsection{Plastic Welding}\label{plastic-welding}

\subsection{salvage components from busted
electronics}\label{salvage-components-from-busted-electronics}

breaking down electronics fast with minimal tools: smash. removing
components with hot air. removing components with a torch, or a candle.
removing components with a pliers and soldering iron

\subsection{how to solder}\label{how-to-solder}

\subsection{Sticks for Electronics}\label{sticks-for-electronics}

\subsection{Sticks for Hydraulic
Machines}\label{sticks-for-hydraulic-machines}

\subsection{Sticks for fluidics}\label{sticks-for-fluidics}

\subsection{chipping Rocks}\label{chipping-rocks}

\subsection{Carving wood}\label{carving-wood}

\subsection{Measuring real time voltages and fluxes with an
Arduino}\label{measuring-real-time-voltages-and-fluxes-with-an-arduino}

\subsection{Measuring electrical transport of
slime}\label{measuring-electrical-transport-of-slime}

\subsection{Finding creepy crawlies}\label{finding-creepy-crawlies}

tardigrades, nematodes, paramecia, bacteria

\subsection{Design a new 3d thing}\label{design-a-new-3d-thing}

blender, 3d printing, wood, plastic etc.

\subsection{Make coloring book style illustrations with minimal art and
computer
skills}\label{make-coloring-book-style-illustrations-with-minimal-art-and-computer-skills}

pencil, pen, inkscape, etc

\subsection{How I wrote this}\label{how-i-wrote-this}

markdown, latex, lulu press, github, github desktop, MOU, macbook air,
various fonts, etc., creative process detailed with library and drugs
and coffee

\subsection{How to color your wood
stuff}\label{how-to-color-your-wood-stuff}

acrylic paint and colored pencils

\subsection{Buying parts}\label{buying-parts}

wire and electronics parts, batteries, cables, some discussion of
alternate sources

\subsection{decorative rope work}\label{decorative-rope-work}

coachwipping, bowline, clove hitch, turks head, monkey's fist
