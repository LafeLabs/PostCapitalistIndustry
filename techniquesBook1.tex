In this chapter I discuss the fundamental techniques that I have used
and plan to use in the near future in the actual practice of what I call
Trash Magic. This chapter will change drastically in future revisions,
and inevitably older editions of the book will look very dated as this
part changes. It is tempting to work on this for years and to withhold
publishing this manuscript until these techniques actually work well and
can be used to make a variety of really nice things. But since a large
part of the purpose of this manifesto is to provide my own guide for my
work, I will plow ahead with some rather immature technology here, and
it will be saved for posterity and will serve as a starting point.

I begin with sticks, because they're fun and easy and do a ton of
things.

\subsection{Finding the Right Sticks}\label{finding-the-right-sticks}

Don't hurt the trees! We want sticks that are no longer part of a living
tree but which have not yet been consumed by fungi and other organisms
which turn logs into dirt. Drift wood is also often too far consumed to
be of use, although this really depends on the drift wood. What we want
are freshly fallen sticks from living trees, mostly. And we're looking
for them to be between an inch and 2 inches in diameter, mostly
straight, with not too large knots if possible.

A lot of electronics projects are perfect for sticks about 1 inch around
and 4-12 inches long, so gathering and preparing these is a good idea.
Load bearing parts for larger constructions should be more like 1.5 to 2
inches in diameter, or bigger in some cases.

Do not be surprised if finding nice sticks is harder than you think it
should be. It can be surprisingly hard! If you live in a humid place
with a lot of rain and water and life, you'll find that sticks get
rotten \emph{very} fast. Sticks that have been out in the weather for a
long time in a dry climate might be rotten while still attached to a
tree in a more humid place. I've spent a lot of time using pine as well
as maple at various times. Pine is pretty soft which is nice for getting
started, it can get frustrating to spend a lot of time trying to cut a
maple or oak stick by hand, especially if you're just trying stuff.

One more thing to mention about pine is that aside from being easy and
fun to work and very common all over the world there is presently an
epidemic in the American west of beetles killing large numbers of pine
and spruce trees. These trees, once dead, are simply a giant fire hazard
that no one wants to deal with, making an unlimited supply of sticks for
the Trash Magician to use should they choose to go forage in that area.

\subsection{Processing Sticks Into
Skeletron}\label{processing-sticks-into-skeletron}

After locating the sticks, you'll want to saw the ends off flat so that
they're not jagged. I often find that it's easiest to gather sticks by
hand without carrying a saw with you when you go out. You can often rip
the branch off by leaning on it with your whole body, be careful you
don't hit yourself in the face!

I also find that for this stage it's good to have two hand saws: one is
a big rip saw with huge teeth that very quickly will cut through wood
but is specifically not intended for metal. The other is a much smaller
screwdriver-like saw with a more hacksaw like blade for easy carry and
use on random materials including plastic and metal. This tool is also
useful for removing knots and branches from your main stick branch.

Once you have your sticks of about the right size, you want to shave off
all the bark. This is done with any of various types of pocket knife,
and I find it useful to have a multi-tool of the kind that is also a
pliers and screwdriver and such for this. Ideally you'll do this where
the massive pile of bark and shavings will be useful for something, like
grinding into sawdust which can be used for a compost reactor. At the
very least somewhere there is already mulch will mean you don't have to
clean it up because it's adding to the existing mulch.

Once your sticks are shaved and cut at the ends, you cut them to size,
shave two flats, and then file the edges smooth(simply for making it
nice, this is not really functional). I generally shave at least enough
flat space to make a nice point of contact when connecting them using
the quarter inch bolts, so at least a half inch of flat space is called
for, maybe more depending on the application. For simple electronics
projects I'll tend to shave the stick down until the whole thing is
about half an inch or maybe 3/8ths of an inch thick.

Finally, I generally drill a series of holes down the middle through the
flat, spaced by at least an inch, sometimes more like 3 inches or more
if I don't need many holes, or with one strategically placed at the
``base'' for an electrical project as a strain relief for the power
cord(mentioned soon!) At this time I drill the holes using a power drill
and a quarter inch bit, generally clamping the stick with a c clamp to
my work bench, which I drill holes in all the time. My bench is a cheap
door on a pair of sawhorses. I also often use a small vice clamped to
that bench for holding the stick while cutting holes. Trying to cut
holes with a drill without some form of clamp is usually a bad idea, is
dangerous and is not recommended.

\subsection{Finding the Right Plastic}\label{finding-the-right-plastic}

I have found that the best plastic for our purposes is LDPE and HDPE
which stand for low density and high density polyethylene. They are
indicated by the recycle symbols 4 or 2, and are mostly cross
compatible.

The easiest source of HDPE for most of us is bottle caps. Standard
plastic soda bottles which are made of PET or similar plastics that I
find more annoying to work by hand usually have caps made of a opaque
material which is typically some color like red or blue. Anywhere
plastic trash can be found, you can probably find these caps. You don't
care how much they have been smashed, but you do care a bit how dirty
they are. You can always grind or cut off the really gnarly dirt with a
knife or file or similar sharp tool if there is too much crust on the
cap. It's generally a good idea to have a small bin filled with these
caps near your work area.

Another great source of plastic, which I use for small electronics work
especially, is the translucent(but not transparent) plastic generally
used for plastic milk bottles. It is also used for various citrous
juices such as orange and pineapple, so if you don't drink milk that's
probably a better bet. Some 1 gallon water containers from generic
brands of bottled water also come this way, and those can be found in
plastic trash piles by various creeks sometimes. I avoid milk bottles I
find that way due to what happens to milk when it's been out a few days.
I drink milk at home and when the bottles are done I try to immediately
wash them out, rip them up, and put them in my plastic material bin.

One way you can get some containers like this if you don't normally buy
them or drink milk is to have a party where the main drink is maitais or
some similar fruity cocktail. You can get orange and pineapple juice in
these containers, and mix them. If you want maximal containers, get the
smallest they sell, and invite a ton of people who like to drink and
you'll have a few containers to work with in a few hours.

Finally, another source for LDPE for very large scale projects like
building boats is traffic barriers, the big orange kind. Don't steal
them, they'll end up in the trash eventually, take those and cut them up
with a hacksaw(they're too thick to cut with a regular knife, although
maybe if you have a giant sword that will also work).

\subsection{Plastic Welding}\label{plastic-welding}

I'm sure there is a way to do this using really free tools, which I do
plan to build. However for now I'm using a very capitalist tool, the
temperature controlled hot air rework tool which I also use for surface
mount soldering. It can be purchased for 50-100 dollars online. I
believe that a hair drier will also work, although the weld process will
be harder to get right due to a lack of continuous temperature and flow
control. I set the temperature to 130 C. If you're using a flame or hot
air gun without temperature control it should be possible to measure the
temperature to target that or just figure it out by trial and error,
which is how I ended up at 130 C in the first place.

The goal with working with HDPE and LDPE is to get it to transition from
solid not to liquid(which you'd use to do injection molding, and that's
well documented on youtube by others) but to glass, which lets you bend
it and weld it but it still has structure. When is it a glass? With the
translucent stuff it's easy to tell: it goes from the milky translucent
color to fully transparent pretty suddenly as it hits the glass
transition which is actually very neat to watch! Obviously all this is
hot, so don't touch it, and be aware that it stays hot after you stop
heating it for a few seconds at least unless you hit it with water or
something to cool it down. Just because plastic is below the glass
transition doesn't mean a 100 degree C thing won't burn you!

As a first weld project I'd say take bottle caps, cut them up, heat them
until they're kind of floppy, and are right next to the, moving the heat
source back and forth between the two bottle cap shards, then when
they're clearly a bit gooey, touch them together, and they should stick,
then heat the combination a bit more, maybe another 20 seconds. Then
when it's clear that they're both gooey and are sticking a bit, get your
pliers or tweezers and start smashing and squashing to get the two to
plastic parts to mix. This is the same basic welding technique that is
used for various food technology like the calzone: the weld joint on the
top and bottom bread in a calzone looks just like the plastic weld
joints you'll make with bottle caps.

\subsection{Salvage Components From Busted
Electronics}\label{salvage-components-from-busted-electronics}

This section is going to be short because right now I still buy a lot of
electronics from the capitalist enemy. As capitalist enemies go,
however, Digikey.com is awesome. There are several companies that sell
electronic components online, with fairly similar prices and selection.
If you want to compare them, the site to use is octopart.com, a startup
company out of Boulder, Colorado which compares all the prices and
stocks of the different companies. That being said, I use digikey
exclusively so that I can have a consistent bill of materials for
everything, which uses digikey part numbers. Digikey can often deliver a
part to you within one day in most of the USA.

As for salvage, the main electronics components I've been salvaging so
far are power supply related. I have found the the best way to get a
power brick open is to swing it by its cord in a huge arc over your head
and smash it on concrete repeatedly. It's sort of like a particle
accelerator, you want the largest possible swing with acceleration the
whole way to get the maximum velocity of impact. It's best to do this on
clean cement with a broom so you can easily sweep up the bits as it
explodes. The plastic case will explode but the components should be
largely unaffected by the smashing. The good stuff in there is likely to
include transformers, capacitors, diodes and bridge rectifiers. Other
things in there will be used more in future versions of this work.

I will leave this section brief since it's very much a work in progress.
I'd rather finish this book and then extend this later than delay the
book while I do the research required to have good specs in this
section.

\subsection{How to Solder}\label{how-to-solder}

The best way to learn to solder by far is to find someone who can solder
and get them to teach you face to face, it's a very physical learning
process. One thing all forms of soldering have in common which I want to
mention here is the need to get the actual metal being soldered hot, not
just the solder. The biggest mistake beginners make is not being patient
enough in heating the other metals that are not the solder. Also note
that whatever is the most massive metal piece will need the most heat
applied, be it by soldering iron or hot air gun.

When I use the hot air gun to solder, it's always with solder paste, and
I set the temperature to 230 C.

\subsection{Sticks for Hydraulic
Machines}\label{sticks-for-hydraulic-machines}

This is another section that has to be a bit of a placeholder in this
first version. Sticks can be used to make various direct mechanical
machines driven by water. Water wheels made from simple arrangements of
sticks should generate electricity to be used in that same apparatus, as
well as to move various belts and cables to move things around in the
world. Water wheels should also be used as a replacement for many
electric power tools, and a high research and development priority is
building a power drill replacement that runs on water.

\subsection{Sticks for Fluidics}\label{sticks-for-fluidics}

Ultimately, the stick technology should have fluidics built into it.
This means channels, chambers, pumps, valves, and electrical/chemical
interfaces. I've done some very crude experiments with this, but since
nothing is really complete this is a place holder for now.

\subsection{Chipping Rocks}\label{chipping-rocks}

We must bring back stone! Not just for decoration but for weight, for
fluids work, for electronics, and for many other applications. Part of
the Trash Magic skill set and tool set must be for simple stone work. I
have been pounding rocks with other rocks and reading a bit about this,
but still have not fully developed the skills.

The one rock skill I have acquired over the years is that if you grind a
rock against pavement for long enough you can polish one side smooth
enough to ``skate'' on, and can push that rock around under your shoe
like a skateboard. This can be amusing, and led to all rock grinding on
pavement being banned at my middle school.

\subsection{Measuring real time voltages and fluxes with an
Arduino}\label{measuring-real-time-voltages-and-fluxes-with-an-arduino}

Measuring voltages in real time should be easy. And yet it's often a
huge pain to transition from doing this in a over-equipped over priced
lab to doing it as a rogue element. The trick is to use the Arduino's
analog to digital conversion, with the new Arduino software's very handy
plotting feature. I generally make a voltage divider with a pair of 10k
resistors from 5V to the Arduino ground, with the midpoint connected to
one side of the thing to be measured, and the other side connected to
the ADC. This is not useful if you're measuring something connected to
the Arduino ground! But if you want to measure something like induced
electromotive force in an inductor it's great, as long as you don't go
over voltage and blow it up. More on this in future versions.

\subsection{Measuring Electrical Transport of
Slime}\label{measuring-electrical-transport-of-slime}

This will be covered in very great detail in volumes II and III, and I
don't expect this to make a ton of sense now, but basically my method
involves putting small amounts of charge onto and off of a capacitor
quickly and observing the voltage response. Doing this fast can create
an audible signal that depends both qualitatively and quantitively on
the impedance, and can also be used to observe nonlinearities through
various doublings etc.

\subsection{Finding Creepy Crawlies}\label{finding-creepy-crawlies}

The real Trash Magic Sticks will have optical microscopy built in, along
with the plumbing to move water with little creatures around to observe
them. For now, I have a very cheap plastic toy microscope that I carry
around on my bike and try to find tiny bugs with. I've seen some
paramecia, some bacteria and the occasional tardigrade shooting by in
the water. Finding water rich in life in Colorado is a bit tricky, but
where it's slow moving and covered in bugs is generally a good bet.

\subsection{Design a new 3d Thing}\label{design-a-new-3d-thing}

Blender! Blender!! Blender is a free open source software package used
for 3d design, primarily for artists. I used to use commercial CAD
software but that is not compatible with the values of Trash Magic so I
had to switch to something free. It turns out that while the
professional CAD packages are better for professional CAD, that when we
abandon the concept of professional engineering and take an artist's
approach that Blender is actually superior. I will not digress with my
own very poor introduction to Blender as I'm still learning the basics
myself. It is, however, going to take a larger and larger role in the
manufacturing and design in Trash Magic as more virtual reality and
augmented reality systems get deployed in the future.

\subsection{How I Wrote This}\label{how-i-wrote-this}

My creative process is to think on things for a long time, then write
fast as a synthesis of that thought. The thinking process combines long
library visits, long walks, long bike rides, Brazilian Jiujitsu and
extensive use of edible THC products purchased in Colorado.

Actual writing generally takes place in a coffee shop, on the couch at
home or in the library, usually in quick bursts of under 2 hours of
work. I did the writing on a MacBook Air purchased around 2012. Most of
the writing took place in the Markdown editor known as Mou. I used
GitHub Desktop to manage the versions and backups on Github, which I
used for my cloud backup as I went. Each chapter is its own Markdown
file. A shell script uses Pandoc to convert all markdown files to tex
files. A file I call main.tex then imports all those tex files as
chapters and provides the format using the LaTeX package ``memoir'' to
generate the actual book pdf(main.pdf).

Illustrations were generated by drawing things in pencil, inking in the
stronger lines, taking a photograph with a small cheap Sony digital
camera, importing the bitmap into the free art software InkScape, and
then tracing into vector graphics. The vector graphics are then re-sized
to fit in a four inch width, saved as .svg files, and then exported as
100 dpi png files which are used for the actual figures in the document.

\subsection{How to Color Your Wood
Stuff}\label{how-to-color-your-wood-stuff}

Wood things should often be at least partly colored, not just for art,
but to code different parts differently so they're easy to follow. I
generally try to have electrical nodes connected to positive supply
voltage be surrounded by red, and the minus supply surrounded by black
or green. I also tend to use yellow for signal and blue for higher
current lines that are not power supply.

I have a set of colored pencils in my work area, which work great on
raw, carved wood, but badly on plastic. I also have a set of acrylic
paints, the smallest and cheapest I could find, which goes on just about
anything.

I believe most of what we make should be colored and that the colors
should always serve both artistic and practical purposes.

\subsection{Rope Work}\label{rope-work}

I will not elaborate on this too much in the first edition, but I will
just mention some rope techniques that are of use:

\begin{itemize}
\tightlist
\item
  coachwhipping
\item
  Turk's head
\item
  clove hitch
\item
  bowline
\item
  bowline in a bight
\item
  dragon bowline
\item
  carrick bend
\item
  figure eight knot
\item
  double figure eight knot
\item
  monkey fist
\end{itemize}
