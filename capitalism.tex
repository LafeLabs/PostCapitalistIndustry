What is best in life?

To care for one another, and to have adventures.

Technology can help us do both of these things, building societies where
all physical needs are taken care of as well as which preserve the
adventure that makes life worth living. However, as technology has
advanced it has increasingly served its own needs. Because it has had
such a powerful overall positive effect on the human condition(in some
material ways), we have allowed the rules of technical progress to
dictate the rules of the rest of our society. In this chapter I discuss
how I view capitalism as an underlying force which drives this process,
creating great suffering for humanity and the rest of the living world.

\subsection{What is Capitalism?}\label{what-is-capitalism}

What is capitalism? This is something that critics of it avoid a lot of
the time to their detriment. If you look up various definitions, it
generally goes something like this: ``Capitalism is the economic system
in which the means of production are privately owned.'' I hate this
definition, and I think it's held back our collective effort to fight it
for the last 150 years.

What this definition implies is that the opposite of capitalism is
someone other than ``the private owners'' or ``the capitalists'' owning
the ``means of production'', and ``economics'' being based on something
other than private capital. I put all these things in scare quotes
because I see them all as subtle weapons to inject hidden ideology into
peoples minds by the very wording of the definition. First of all, the
anarchist rejection of capitalism rejects ownership of minerals, land,
and machines. So any definition that talks about ``who owns what''
should already be rejected by the anarchist, and we have already ceded a
major point by allowing this definition to stand at all unexamined.
Capitalism is a system in which some people, called ``owners'', claim to
have power over certain things which they claim the right to carry out
by force if needed. Capitalism is a system in which a military state
exists which both feeds of the system of privately owned extraction and
enforces the power structure that governs it.

The ``means of production'' is also a problematic phrase. While it is a
bit ambiguous, I see this phrase as at least potentially implying that
this the ``means'' is some sort of fixed infrastructure. The implication
is that ``the means of production'' is a thing that exists outside of
economic systems, which can be controlled by any of various types of
government or state. This is false. The very structure of ``production''
in today's society is what I would call capitalism. The Soviet system,
the various fascist systems, ``democracies'', dictatorships, monarchies,
I would say every single one of them is capitalist. They all have this
basic structure of military power creating a monopoly of force that
protects a vast system to extract mineral wealth and destroy it as fast
as possible by constant threat of violence. To me calls to ``seize the
means of production'' sound like calls against a king to go seize the
palace and tell the king what to do but to keep the palace and king in
place. It's the same system, with slight changes. So to let the
capitalists define these ideas gives them a victory before a debate even
begins: it allows that the existing ``means of production'' should
continue to exist without discussion. A true challenge to capitalism is
one in which the very concept of production is reinvented. It means
building industrial technology from the ground up around different
values.

Another problem is with the notion of ``economic system''. I would argue
that economics is again a part of the intellectual descendent of the
basic idea of the One God of monotheists. There is a Universal
Heierarchy that exists, which allows numbers to be used to assign value
to things. Human value becomes a number, always either less than or
greater than or equal to any other numerical human value. Part of
rejecting the basic ideas of capitalism is to reject this hierarchy cast
down from God. But to even use the phrase ``economic system'' again lets
capitalism be defined in a universe in which nothing other than
capitalism exists.

Indeed in some of the definitions I've found online they even add
phrases like ``as opposed to State ownership of the means of
production''. In other words the supposed definition of capitalism used
by most people is not a definition of capitalism at all, but a clever
propaganda piece that creates a world in which the alternative to
capitalism is another type of capitalism which is re-cast as the
Socialist Enemy. Since I consider all the Soviet style ``communist''
countries to be capitalist in their philosophical worldview, I find it
not surprising that they hold the same warped view of this false
dichotomy. The communists can point to ``capitalism'' as their enemy,
where ``the ruling class'' ``own'' the ``means of production'', rather
than ``the dictatorship of the proletariat''. When this becomes a
nightmare like it always does and destroys the environment even worse
than ``capitalism'', people on the right say ``I told you so'' and
people on the left say ``it will be different next time! it's all
Stalin's fault!''.

So if we really want to move beyond capitalism, criticisms of it need to
start trying to really see it for what it is, and see just how far the
viral ideas about God that underly it have wormed their ways into the
very language we use to describe it.

I will give capitalism the following definition:

\textbf{Capitalism is a system of belief in which numbers are used to
denote all value.}

That, I believe, is the heart of the matter. And it points to why
experiments like the USSR have ended up having problems so similar to
those in the western capitalist world. In a word, money. Money is not
just metal or paper or faith in a government, it is the idea that a
number, specifically an integer number(money can usually be subdivided
but only up to a point) can be used to denote all human values. This is
why I believe the concept is so slippery, and so hard to break out of.
You can replace dollars with time dollars, bit coin, gold, silver, bags
of salt or gold-backed e-dinar and it's really all the same thing:
numbers. Integer numbers. As long as there is an exchange rate between a
system of value and an existing currency you have not really broken free
of the current system.

And what is money? The purity of numbers has proven to be incredibly
powerful. Users of the number based values have literally moved
mountains with the power they have been able to deploy using money. In
particular money based values have been excellent at several things,
some of which are good but most of which are bad. I will now explore the
nature of money more specifically.

\subsection{The Nature of Capitalist
Money}\label{the-nature-of-capitalist-money}

Our currency is based on two things:

\begin{enumerate}
\def\labelenumi{\arabic{enumi}.}
\item
  suffering
\item
  and minerals
\end{enumerate}

Turning minerals and human misery into numbers is capitalism in a
nutshell, and is the basis of our monetary system.

Capitalism is an industrial system in which all value is based on human
misery and minerals. By creating misery, some people use threats of
violence to control land. They use more minerals, fire, and misery to
create minerals ordered with a precision based on their belief in
violence and control through military order. The threat of inflicting
misery using military technology(not only is our technology military,
our concept of military is based on our technology as well, and both are
based on the One God beliefs) is how some people known as capitalists
claim ``ownership''. Ownership is a complex network of violent threats
which allow threats of future misery and benefits paid from past misery
to be added up numerically, building a ladder of power down which the
physical benefits of mineral wealth slowly trickle, with the most
landing at the top.

Any proposal to reform capitalism that maintains concept of numerical
adding up of suffering and minerals is just capitalism with a new mask
on. True reform means finding a set of moral values that informs
technological figures of merit which are based on human joy, adventure,
hilariousness, beauty, or other things that actually have positive value
for everyone, and then re-builds our whole concept of what it means to
have a technology up from scratch.

To repeat: to attempt to reform capitalism while continuing to use any
of our current technology at all is a lost cause. The ideas of
capitalism are built into the position of every atom in a modern
technical artifact. If you want a world without capitalism you must re
order every atom, completely re design how atoms go together from the
bottom up. And in building this it makes sense to acknowledge that
300-400 years of industrial capitalism gave us the gift of minerals,
which we can now live on forever.

Every atom. Every atom changes in how it relates to the whole. Same
physics, same atoms, but new ordering principles, breaking out of the
military design concepts. No more are the ideal shapes always planes,
circles, and perfect grid arrays of objects. No more are tech artifacts
locked into a centrally controlling clock that tells them when to work
and what to do. No more is there a wall between engineer and customer,
where some things are known and some are secret: all information on
construction is physically encoded in the artifact, and updated as more
edits are made, even if the user does not document(data stream into the
dataverse).

\subsection{Capitalism as Religion}\label{capitalism-as-religion}

Capitalism is the hidden religion. It does not admit to being a religion
and its believers(at this point almost all humans) do not realize they
are in this religion but they are. Even members of various other
religions decry people leaving their flocks for the ``secular'' world
but won't directly name this as a competing religion. But a religion it
is, complete with odd beliefs of all kinds.

In my observation, the beliefs of capitalism include:

\begin{enumerate}
\def\labelenumi{\arabic{enumi}.}
\tightlist
\item
  Private property is sacred
\item
  All value can be added up using numbers
\item
  All value must be extracted from the Earth or from human misery
\item
  Human society is described by something called an ``economy'', which
  is a system for laundering mine products and human misery into
  numerical media of exchange
\item
  Hard work is an intrinsic good
\item
  Our world can all be described by a giant hierarchy, people, animals,
  objects, gods, ideas, all are always ranked and this ranking is
  ordained by the highest authority, whatever that is.
\end{enumerate}

I believe that number worship is an underlying hidden religion that is
integrated into all other modern mainstream capitalist religions. What
is monotheism? It is the belief that there is only one true god. But
this implies that you can count gods. That is the underlying assumption.
It separates parts of the universe that are god from non-god in a rigid
way, breaking up gods or potential gods into discrete numbers that one
can count, rank, and ultimately then put one on top of all others. From
this we get hierarchy of all kinds down through the ages and all the
horrors of capitalism. But if you are a monotheist note that your One
True God is almost certainly also a universal part of your world. So
what makes you believe you can count gods? This other, hidden, religion
that is required to phrase the questions and answers about your god
using numbers. So do not take my attacks on the structure of industrial
monotheism as an attack on your One God--I do not deny your god, merely
your ability to count gods.

That being said, I do think this counting has led to other problems in
industrial monotheism which must be combated, namely patriarchy.
Monotheistic religions have a strong tendency to extend the counting
hierarchy from their bearded man-god down to all Things, building an
instant patriarchy into their world view. Don't do that!

Number worship, the belief in numbers as a superior picture of reality
than other models. BBC documentary on history of numbers is actually
blatant capitalist religious propaganda nonsense. Vietnam war, big data,
the very word ``rational'', always the assumption that number-based
ideas are superior than ideas not based on numbers.

\subsection{Professionalism: A Capitalist
Cancer}\label{professionalism-a-capitalist-cancer}

I am against professionalism in all forms. Professionalism divides us.
We have split up philosophy, physics, chemistry, biology, design,
manufacturing, theology, art, and technology, and very much to the
detriment of them all.

I'm against engineering and design as professions. While specialization
can be useful, I believe our society has created a soul-less
techno-priest class which is evil enough in its very nature that
technology needs to be re-built from the ground up outside that system.
If your technology needs the techno priests to function, it means your
technology is bad and needs to be replaced. If it needs extraction of
raw materials from the earth or any control over large tracts of land in
a centralized way to function it is bad technology and needs to be
replaced. If it requires secrecy or proprietary control of information
and use it is bad technology. If it can't function without capitalism it
is bad technology and needs to be replaced.

Specialization is fine up to about 100 people then it is a luxury for
special projects. If you need someone who makes up less than 1\% of the
population to do something your technology needs a reset and it is bad.
Our goal is total freedom for 100 people.

We need to start over from scratch and build a technology without the
existing techno priests which can be built and maintained by anyone with
the desire to do so, using waste streams of the old system. This has to
happen in thousands of parallel tracks in many different fields of
applied science and technology. I will focus on the parts relevant to my
area of expertise: applied physics.

\subsection{Capitalism Stifles
Innovation}\label{capitalism-stifles-innovation}

Part of what has led me to write this work is my frustration as a
professional scientist with how capitalism has, in my view, held back
scientific, technical and cultural innovation by decades if not
centuries.

There are several aspects of capitalist ideology which have had
devastating effects on science. The first is the obsession with novelty.
This is probably the largest problem, which I would say has gotten
progressively worse as science got more advanced over the last 100 years
or so. The problem is that in order to be seen as a success in science
you need to prove that what you did is really new, and that newness
takes priority in value over almost everything else. What this does is
create a very broken ladder of importance of things to study. If you
have the choice between two experiments which both show the same
science, and one involves just seawater, dirt, and a mobile phone, and
the other involves a 1 million dollar machine, a trendy new molecule,
and some advanced math using a new computer algorithm, the latter is
considered vastly superior. And this is based on the ideology of private
property, even when legal intellectual property is not involved. Even in
the public domain, when a researcher publishes a sufficiently new thing,
that thing is attached to their name, and can be turned into real
tangible monetary value.

All the elements I describe in the example above should be called out
for causing problems with science progress. First of all, the use of
expensive machines. This not only makes sure there is a barrier between
the work of the lab scientists and the general public, it usually
increases the distance between the researchers themselves and the
subject matter. I believe that the purpose of all science is to create
the closest possible link between the human mind and the world we live
in. The more expensive your machine, the larger the barrier between mind
and world. Expensive machines are great for building capitalist
jobs(I've had these jobs!) But this is at cross purposes with what
should be the goal of simplification. To eliminate a machine is to
eliminate a high paying technical job, which hurts us as workers in
science. Thus the incentive is opposite of what we want to do, which is
always cut down the the size and number of machines needed to interact
with our world.

Another element of the problems I've listed here is the ``trendy
material'' problem. That is, science is strongly biased in favor of
newly ``discovered'' materials over those we all know and have access
to. This is created by capitalist ideology because we all need to try to
own the property, both legally and intellectually, of ``new'' things in
order to get the fame required to advance in our careers. If you prove
that ``your'' substance has a different chemical structure than any that
someone else has studied, and publish something not very impressive, you
can get famous, and name the molecule. But if you do something
impressive, but not really new, on something common like tap water or
ground up moss or a soda can, you have to call it ``educational
demonstrations'' and will not be taken seriously in high level research
circles. But again, this is creating an incentive to do the opposite of
what is good for science. Someone who interacts with tap water or
pavement has a connection to much larger fraction of the world than
someone who interacts with an obscure form of soot made in a special
chamber that only exists in their lab. If our goal is to connect our
minds to the world as well as possible, it's always better to follow the
most common elements of that world, then things we find around us.
Capitalism pushes the researcher away from those things both because of
the need for novelty and also because the more obscure a molecule is the
more likely it is that a capitalist can make a profit on it. A product
based on a simple recipe with tap water and gravel is worth infinitely
less money than one based on a complex and expensive process.

The ephemeral concepts of ``ownership of ideas'' above pale in evil
compared to legal intellectual property. This could be a whole polemic
work of book length on its own but suffice it to say that the corrosive
effect excessive patent and copyright are now so severe that anyone
who's worked at all in science in the last 10 years is already pretty
upset about this issue. Even those who claim to support the system agree
that it's now so far beyond even the twisted intent that originally
existed that they are against it in its current form. However, for the
record, my position in this work is that it is pure evil to claim the
concept of ownership over science or technology. The scale of the evil
is partly escalating as the technology becomes more personal. As our
technology becomes more a part of not just our lives but our selves, we
find corporations claiming to legally own parts of our lives and even
our bodies with their patenting of genes both in humans and in our
various bacterial neighbors we carry on our bodies. Eventually, the
property ideologues will, if left unchecked, build a world where humans
are all owned by a consortium of corporations, where we are all
literally the property of corporations and machines. Science fiction
warns of the possibility that a ``rise of the machines'' will cause us
all to become slaves to artificially intelligent machines, but I would
argue that AI is not needed for us to become slaves to machines:
humanity is in the process of enslaving ourselves to non-intelligent
machines.

I touched on the problem of professionalism already but I need to
elaborate on this in the context of science specifically. We have always
claimed in philosophy and science that unification is a goal.
Unification of electricity and magnetism into one theory and then the
weak force in with that are all seen as great triumphs of physics.
Bringing all the atomic elements together into a single unified periodic
table is rightly seen as a great triumph of chemistry, etc. But in
modern applied science we find huge incentives in the opposite direction
of unification. Because we are all forced to carry out science in the
professional system, and there are never enough professional positions
to go around, those with the good professional jobs must all jealously
guard our positions. This means a biologist who can do good physics or a
physicist who can do good biology are both potential threats to each
others' jobs. Whereas the biologist who creates an even more obscure
form of biophysics that gets its own whole new department is the most
powerful of all: the unique specialist who owns their field entirely.
The highest salaries and most honored and secure positions will go to
those who do the opposite of unification. And sure enough, the last few
decades have seen a proliferation of tiny sub-fields with their own
jargon no one else can read in all fields of science. This has coincided
with the rise of extreme market ideology since the 1970s which drives
universities to behave more like businesses and research departments to
behave more like marketing departments. The corrosive force of
capitalism has inflicted a sort of Babel curse on all science, making it
impossible to talk to each other anymore.

This concept of unification applies in particular to building the tools
we use for science. The most useful tools are the most universal: razor
blades, tweezers, optical microscopes, or pliers. And yet no
professional scientist can make a living selling any of those, so we're
not incentivized to make more tools like those. We can make them for our
own use in our labs, but capitalism directs those types of tools to be
made by the cheapest possible labor, so building them is avoided by the
professional classes. Conversely, the tool which only does one thing
extremely well can be a perfect monopoly on that thing, creating a large
markup and building a comfortable place for the professional. Again this
is a case of capitalist ideology constantly pushing us all to build the
opposite tool from what would benefit our fellow scientists or the rest
of humanity.

These claims are just claims when stated in a a manifesto like this. I
state them without extensive proof because the proof that abandoning
capitalism can push science and technology forward much faster has to be
by example. We must actually go out and do this, build science and
technology up from scratch on non capitalist principles, without
professionalism and without property. Ultimately this ends up looking
more like an artistic movement(for which a manifesto would be a normal
part of the creation process) than a part of science. Trash Magic will
take many forms in the future, but its initial form will indeed be that
of an artistic movement, because that's the simplest way to build things
while casting off the old figures of merit used by engineers and the
rest of the technocratic priesthood.

\subsection{Death to Capitalist Math!}\label{death-to-capitalist-math}

Math is not objective reality. This is obvious to most people who don't
do math, as well as to most working mathematicians, but it's an
amazingly popular belief among technocrats. Math, like any other model
built in the human mind, is a sort of reflection of the world. A very
powerful one, yes, but still just a part of our minds, and like any
other model, there are choices we made to get where we are with math
which could have been made differently.

The example I'll give here is a paradox that I find particularly
interesting in terms of what it tells us about hidden ideologies.
Mathematicians call it the Banach Tarski paradox, and it generally
arises in parts of the math curriculum concerned with point set theory.
Never mind exactly what that is, it's something usually taught in the
late undergrad or early grad level in pure math(as opposed to applied
math which is not concerned with these issues).

What this so-called paradox does is create a way to construct two
spheres of points from the points in one. That is, all the points in the
first sphere are re-arranged in such a way that those same points make
two spheres of the same volume as the first.

\subsection{Why Now?}\label{why-now}

Now is the time for \emph{drastic} change unique in our history.

Why now in particular?

Both the positive and negative sides: danger to humanity is imminent,
but also opportunity is greater than ever before because of the vast
mineral wealth that is everywhere and a critical mass of processing and
communication technology. Marx was about 100 years early, and didn't
have access to the information or materials we do today. Globalization
and Capitalism really have literally sewn the seeds of their own
destruction, by creating seeds for millions of new societies by
spreading mineral wealth everywhere around the globe.

The very destruction of capitalism focuses us on the better future in
several ways. For one thing, the sections of society most exploited or
crushed by capitalism are often also those closest to the massive waste
and destruction streams of the present system. Often the poor and
dispossessed live near dangerous waste which also contains what could be
priceless mineral wealth if we had the technology to bring it back.
Wherever you find the most oppressed people you will also usually find
the most ruined land with the most material waste. Just like the people
our economy casts aside, these materials often exist outside the
ownership system, they are claimed by no one and valued negative or not
at all by our economic system. But this creates a potential opportunity
to build very rich new forms of industry that exist without ownership or
money: built by people who no one pays, made from materials considered
``toxic waste'' by the ownership society, and given freely to a
community who also owns nothing undermines the entire structure of the
existing system.

This connection between the people and the materials cast aside is what
Trash Magic is really about. People who's time capitalism does not value
can use the materials it does not value to truly work magic: to build
great works of art that we can live off of using the powers of our
minds.

\subsection{Purpose of this Book}\label{purpose-of-this-book}

This book is a manifesto. That is, ``\ldots{}a public declaration of the
purpose, principles, or plan of action of a group or individual.'', as
it's described on manifestos.net.

Note that novelty is not my goal. I believe that the obsession with
novelty in applied science is a toxin of capitalism and that by ignoring
where ideas come from and using them as needed, with no expectation of
novelty that much faster and better progress can be made. This work
comes from the heart and mind of one person but none of that comes from
just me: I assume everything I say here has been said elsewhere and that
I've been exposed already to most of what I present here, in various
forms, in books I've read or from people I've talked to.

\subsection{Paths Out of Capitalism}\label{paths-out-of-capitalism}

I'm against the machine. That's what this is all about. I hate
industrialized society, and I resent that the good products of it are
used to hold us all hostage to the totality of The Machine. The military
machine, the capitalist machine, the consumerist machine, the extraction
of raw materials machine, the political machine, all of it. We're told
that if we it's all or nothing. Don't like nuclear bombs? No vaccines
for you. Sick of the Internet giants controlling your life? Well, I hope
you like writing letters by hand, asshole, you must be a Luddite. That's
the message over and over from the mainstream of society.

I challenge all that. I say that the course of the last 300 years of
industrial development has not been just fixed by some immutable laws of
nature but has in fact been the product of decisions made which could
very well have been made differently while still learning how the world
works and how to make useful technology to better navigate that world.
