\subsection{Technology Should be Slimy and
Dirty}\label{technology-should-be-slimy-and-dirty}

Look around you. We are bags of salty dirty water, and we are surrounded
by mud and dirty rocks on all sides. This is the world we live and grow
and thrive in. It's how our food grows, it's how our waste is disposed
of, it's how we get our raw materials and how we dispose of our
``trash''.

And yet this is not how our technology is.

Our technology is, instead, obsessed with the clean and ``pure''. It is
obsessed with order, with perfect rows of things, with straight lines
and perfectly geometric circles. The very structure of all our
technology represents our worship of numbers and math and military
order, as well as of mining and minerals.

I will go into more detail on this later, but I believe the structure of
the modern micofabricated circuit is a product of the white supremacist
ideology of the far right lunatics who started Silicon Valley. They
were, like all of their kind, obsessed with ``purity'', order, and
forcing everyone to march to a perfectly timed clock. This is borne out
in a machine architecture which they pretend is a product of some kind
of technical evolution but which is just as much a function of their
capitalist religion as the rows of decorative stone columns they out
outside their buildings of power.

If, rather than Evil Machines, we want our technology to be more human
and more life like, it should resemble what we see around us in the
living world. This means it should be largely filled with and immersed
in dirty water. And should be capable of moving fluids and gasses around
at around atmospheric pressure, with simple circulation systems.

Another key distinction of living systems is that they do not
distinguish between material transport, data transport, energy and
electrical transport. All of these involve the flow of ions and various
big molecules through fluids.

Our non-living technical systems crudely split these functions off from
one another.

\subsection{What is Fluidics and why do we
care?}\label{what-is-fluidics-and-why-do-we-care}

One type of magic that must be wielded if we expect to have a decent
life is potion making. This means mixing fluids, moving them around with
pumps, compressing them, running electricity through them, and also
doing things with gasses of various kinds. This is needed to efficiently
compost waste at a high speed safely and to build up plant growth
infrastructure fast for food production. It is also where novel
chemicals and various life saving drugs come from.

\subsection{Light Magnification and
Projection}\label{light-magnification-and-projection}

Build lenses into sticks that can magnify and project the magnified
image. With the ability to project microscopic things, it should be
possible to do real time display with millimeter scale vibration of
fluids and suspended objects in fluids. This could be the free analog
scope I've been looking for.

There should be both a projection system and also a system for direct
projection onto the eyeball of the user in a goggle configuration.
Essentially the standard microscope configuration but with better
ergonomics.

\subsection{clean water}\label{clean-water}

clean water is CRITICAL, need to research existing tech here, talk about
alternatives.

\subsection{Rants about free transistors vs.~Fascist
Transistors}\label{rants-about-free-transistors-vs.fascist-transistors}

\subsection{from the blog:}\label{from-the-blog}

The channels in the reactor should not just be able to take water, water
should be in them most of the time. If the reactor is just barely
partially submerged in water, it can get power from waves and tide or
current in a river or stream, both for generating electricity and as a
direct source of hydrostatic pressure to push materials from one
``side'' of the fractal reactor to ``the other''. I put scare quotes
around side and other here because it's not really going to be arranged
as a simple in/out machine, but will have many channels that direct
materials around.

Having the channels be filled with water means that robots can be made
to be neutrally buoyant or close to it, and propulsion along the
channels of the system can be caused with pumps rather than some kind of
motor on the actual robot. Thus a ``dead'' robot can be directed to a
location in the system without being powered up at all, then can park on
location where the work is to be done and absorb energy from flowing
water to use to do mechanical and/or chemical work on location.

The motion of the water can be used to generate electricity which also
splits water into gaseous hydrogen and oxygen. Pumps could be used to
liquify the H2 in some cases as a more dense energy storage medium.
Having oxygen available all the time is useful for chemical reactor
chambers because you can use oxygen plasma to clean surfaces in an
aggressive way to ready a chamber for a new process. And H2/O2 gas torch
can be used for melting glass and various simple welding and heating
tasks. Finally the re-uniting of the H2 and the O2 will create pure
water which is needed for humans and plants.

Just the wave or stream powered desalinator and burner that can run with
no intervention is a technology very worth building ASAP. Running water
can turn a turbine that makes DC AC power which is rectified and
smoothed to DC, then drives electrolysis, saves it up until a gas
reservoir fills, then an electric spark from a step-up transformer from
the generator fires, reuniting the O2 and H2 to form pure water, which
falls into a water reservoir which can be of arbitrary size. Over time,
this could sit for months building up fresh water very slowly, making it
always available in that location. This must be built! If it's possible
to use the metals in gas powered cars to build fuel cells, this could be
the basis of a primitive hydrogen economy.

In addition to power, locomotion, hydrogen and oxygen, water can provide
many other useful chemicals, many of which are unwanted contaminants
which will be removed from the water over time. Ocean water in
particular can yield vast chemical wealth from the trash found in the
ocean as well as energy from the tides and waves which will be
substantial in many places. Unlike a river, in the ocean one can build
out in a 2 dimensional area and always have both energy and materials
that scales up with the building. Why is this relevant? Because this is
the future of many coastal cities. As our climate shifts and sea levels
rise, many coastal cities will be flooded. Some will be rebuilt higher
up with landfill, with water pumped and managed like in Amsterdam. But
many will inevitably be abandoned. This has the potential to be a
environmental disaster of its own right, as the many toxic chemicals of
a modern city suddenly can flow naturally out of the city.

However it's also a huge opportunity for the trash wizard to build the
means of production. As the refuse of a dead city first starts to float
freely, if fractal reactors and rumbles of robots etc move in fast
enough, the entire mineral wealth of the city can be reclaimed into the
new system. Vast, scalable, production of fresh water will enable a huge
canopy of vegetation to be grown above the site, using the buildings as
a lattice for upward growth to create a green paradise where humans will
move from place to place via skylines of various kinds, some manual and
some powered by electricity or wind or tide, as well as boats of various
kinds. Storms and floods should be able to be captured in terms of both
energy and materials, building up huge amounts of pumped-uphill water as
well as hydrogen and oxygen. In some cases it will make sense to run the
reactor as the storm hits to use very high power levels temporarily to
complete a large industrial task quickly with the surge in available
energy.

I want to make one more point about reactor chambers in the fractal
thermoplastic system. It should be possible to change the shape in situ
using the various assembler robots that can carry heated reshaping tools
around. It's probably easiest to do this in either air or vacuum or some
inert gas, but that can always be done because vacuum and gas pumps are
ubiquitous in the system. I think we also need ball valves that are
driven by the magnetic motors. But if you want to do an edit, you can
always use fluids to move your robots into place, move them into a fixed
position, evacuate the chambers, run flown water through a neighboring
location, generate power off that, create heat and motion and use the
robots to use melt tools to weld, cut and add plastic to make new
shapes, motors and tools.

Three dimensional mater that is made from pure trash and ambient energy
which has the power to edit itself. That is what this technology is
pointing toward and I'm not seeing anything that stops it from
happening. Unlike the Drexler/Merkle model of nanotechnology, this
fractal approach does not start out nano. It will be useful immediately,
and will continue to be useful(indeed revolutionary) even if the
nanotechnology never works.

One more point I want to make is that this technology allows for all
sorts of scalable bio-reactors to be built, so that all the advances of
biotechnology can be leveraged into our free infrastructure. Drugs and
other very useful chemicals can now be synthesized by genetically
brewing with genetically modified organisms in the same way that beer is
brewed. The ability to scalably manipulate liquids should allow for both
the development and deployment of genetically modified organisms to brew
useful chemicals. The ability to make drugs and vital nutritional
supplements in this factory is critical, and again, is a technology that
on its own is worth developing even if the whole rest of this plan
didn't work. Given the fractal nature of our technology and the fact
that it's immersed in water, organisms of all kinds can flow in and out
of it, and be used for many purposes. A symbiotic relationship can and
in fact must be developed between the smart matter technology and the
surrounding ecosystem. Looking just at the flow of atoms, there are lots
of atoms that the human body could use to live in ocean water. Could a
system like this produce a food of some kind just from flowing seawater?
Maybe. With enough energy and time(both of which a patient society has
plenty of) it might be possible to live 100\% off the ocean.

Obviously human and other animal waste must also be processed in this
fractal reactor system. Again this is a source of incredibly useful
atoms. Just the methane that leaks off from solid waste is like gold for
early work on the nano assemblers since it can be used to make 3d carbon
nanotechnology electronics and described above. A living system should
be able to digest waste much faster and more safely than the current
systems where the only living thing is the one target organism that does
the digesting. And of course the reclamation of chemical wealth in the
form of drugs and minerals will also be of huge monetary value in the
old economy quickly, creating a ready source of central bank debt wealth
for the community who lives off the Reactor in the sea cities.

\subsection{another blog post:}\label{another-blog-post}

In reference to the previous post, on the fractal factory, I have been
thinking more about the materials to use. I have been thinking about
silicones for everything. But why? The basic principles in the previous
post will work best with some type of thermoplastic, because that can be
done easily with the free metal etching tool to make moulds. heat,
pressure, and a scanned high voltage welding tool can make layers from
any of these types of material. A scanned heated head could also just
make shapes, and of course there are 3d printing techniques.

Here is a list of some thermoplastics:

https://en.wikipedia.org/wiki/Thermoplastic

PLA, nylon, ABS, PMMA, HDPE, LDPE, and even teflon can all be used!

Also, the correct way to fabricate arbitrary 3d objects from many layers
of thermoplastic is as follows. First, metal parts are made by a
combination of machining, 3d printing, and laser cutting. These could
also be chiseled manually with robotic vibrating parts. High mechanical
impedance vibrating tools with cutting steel and diamond cutting tools
chipping like a beaver's teeth, being moved around with scanned x y and
z motion could cut arbitrary curved surfaces into metal.

All these various metal tools can get picked up by robotic arms and
moved around and pushed into a target material, where the metal tools
are heated to the correct temperature to reform the plastic. This allows
tools to fall over a range of size. The same infrastructure can be used
to deploy a several cm tool to cut a several cm channel in a big block
of plastic, and then micron scale tools also press patterns into other
smaller locations. This scaling of tools means that the fractal nature
of the finished product is represented in the fabrication method.

3 dimensional mechanical thermoplastic lithography. Robots must be built
which can carry out tasks with heated tools in various locations and
scales.

The task to start fabrication becomes clear. To start off, metal parts
are salvaged of various shapes and sizes, heater wire is wrapped around
them with thermometers and temperature regulation, put it on the end of
a stick of some kind, and impressions are made in different
thermoplastic trash from cars and similar household junk.

Going back to the various motor designs, it should be possible to design
the lithography tool set for making the bits out of various
thermoplastics to get all the basic motor types to work. The path to
making all these things work keeps getting clearer. With fluid
transistors that can be fabricated with channels in plastics, simple
logic gates can be made, as well as various types of control and
amplification. With robots that deploy all shapes and sizes of thermal
press tools, it should be possible to make arbitrary 3d fractal channel
structures. This means we have transistors, motors, and pumps. With the
high voltage generator technology working it should be possible to build
it out of the fractal trash foam. Wires can be pulled through channels
by rolling robots of all sizes, and they can go into the liquids to
connect them. Wires separated by a small fluid chamber can make a high
current transistor.

So the elements that are coming together in this vision are: Rumbles of
robots can rip a car apart and make assemblers, motors, processors,
energy storage devices, high voltage generators, high current
generators, pumps, compressors, and finally more robots. Plasma and
chemicals being pumped and mixed in the plumbing allows for arbitrary
synthesis of needed chemicals, machines, tools, and electronics.

Just the fractal robotic fabrication of thermoplastics found in trash is
worth pursuing as an isolated technology. Add that to motors! Chemistry!
CVD! wow! If this works based on even just using piles of trash and
heater wires with sticks used by hand, it can be used as a demo for
getting funding and doing useful fluidics.

\subsection{and another:}\label{and-another}

One of the layers in the trash wizard technology is microfluidic
channels that can move ionized gasses around. The implications of this
should be examined. Ionized gasses are an essential tool in modern micro
fabrication. With plasma plumbing integrated into everything, all the
tools of standard microfab are also integrated. Thus it could be
possible with micro sized plumbing and materials moving vibrational
motors and pumps to fabricate electronic components, move them around,
physically re-assemble them, and break them down and reform them into
other devices.

This can be a path to scalable synthesis of nano devices. Carbon
nanostructures could be built and also positioned by flowing gasses that
are used for CVD synthesis through a channel that then defines the
position of the structure once it's fabricate. Perhaps use of a standard
CVD nanotube recipe where the catalytic particles are positioned using
the particle movement tools described above, and then where more
vibrational motors built into the channel are used to create junctions
of various kinds to create arbitrary nonlinear circuits.

In order for all of this to work, we first need to have a scalable
fabrication system for making micro fluidic channel circuits with pumps
built in that can work for liquids, gasses, and tiny solids. I believe
that a great starting material for this is PDMS. Unlike some other
silicone choices that could be used for the fabrication, PDMS flows with
time, temperature, and pressure. A scanned probe that deforms metal
could be used to get a very high resolution shaped surface. This can
then press a sheet of PDMS up against a flat metal plate, and with
applied heat and some time a perfect copy of the metal should be made in
the PDMS. This could be done to make a sequence of layers, which can
then be stacked and pressed with moderate heat and time to join them
without destroying the shape of each layer too much. Many layers can be
added up to make a 3d network of channels. Some channels will have
conducting fluids in them that can control the potential of various high
voltages. High voltages are used, with feedback, to drive various floppy
and also springy bits of silicone. These flaps, fins, spines, rods,
pushers, and membranes will act to move fluids, solids, and gasses
around the network.

With the ability to move anything to anywhere in the 3d network of
silicone, we can connect the whole cube to various input gasses and
liquids to do chemical synthesis. This can also involve various types of
chemical and physical vapor deposition on solids that are moved around,
connected to high voltages, plasmas, etc. So arbitrary nonlinear
circuits should be able to be built into the superstructure using a
variety of technologies, which grow and shift over time(while keeping
the PDMS matrix unchanged). This is very powerful, and justifies a
potentially laborious and slow build up of the 3d silicone matrix. If
one fabrication run can yield many generations of technology, it can
afford to be slow.

Perhaps the first practical step is to build the high voltage generator
that is part of trash wizardry. If we had high voltage, we could start
playing with electrostatic motors and also would have a source of high
voltage for trying to do scanned probe nano machining of metal surfaces.
The way this generator might work is a mechanical oscillator will be
driven by the standard philosophy engine magnet/coil driver, and will
have appropriate metal and insulating pointy and not pointy bits that
move charge from ground to some isolated electrode. Experiments with
these types of oscillators could begin at any time, and need almost no
funding, just time to work on them. With high voltage working, motors
are needed, as well as welding tips. Then scanned probe weld fabricators
are made which are driven by simple electrostatic motors and can make
molds for PDMS. From here, a system is built up for making arbitrary
networks, then chemical and physical synthesis is built up.

Another important point about this whole technology system is that it
should be fractal scalable. That is to say, there can be channels and
pumps and switches and so on that are as big as meters across in some
massive factory block down to 10's of cm, cm, mm, microns, and finally
down to just a few nanometers. With motors that also scale all the way
up to huge movers of many kg solid objects using coils and magnets down
to electrostatic drivers that move plasmas around at the 10 nm scale to
fabricate new nano materials, it should be possible to take in raw
trash, rip it apart fractally, move destructor robots around as needed
to keep it all apart and sorting it, and at the end the scale is
nanometers both of the input and of the output, which are nanoscale
electronic materials. These materials can be moved around on other
``sides'' from the inputs, scaling back up as needed to create arbitrary
technology with nano structured material.

Can this all be built up from cars? I believe so, as long as some rubber
can be found in cars that can work for the synthesis of the
infrastructure. That probably won't be PDMS, but I believe tire rubber
can be made to work with some experimentation. Fart gas from various
animals and hydrogen and oxygen from electrolysis can make a fair number
of initial processes to start building up technology with. This may be
the future of domestic cattle: as a source of methane to use as a
process gas for fabricating carbon nano electronics. One can imagine a
giant dome tent which channels the methane away from some cows and pumps
it into the nano assembly. Rotting turds can also be used to source
methane, which can be huge piles of accumulated dog poo in urban areas.

I'm suddenly contemplating how the many elements should communicate. I
think high voltages turned on and off could be used to create electric
dipole sources that can transmit at low frequencies and speeds very
easily. For very low data rates, as a constant baseline, all the
elements of the trash wizardry could communicate using high voltage
oscillators controlled by high voltage liquid transistors.

Also note that it should be possible to quickly do science to
investigate new ways of doing things. For instance liquid transistors of
many different liquids should be able to be very rapidly investigated
and instantly deployed after discovery.

It appears that it should be possible to create a totally scalable
nanotechnology system that can break down old cars and create arbitrary
new nano electronic and mechanical materials, which can create still
more factories, made from the same materials which absorb large and
larger numbers of cars, etc, creating exponential consumption of cars
until there are no more cars.

all the stuff about the evils of the microprocessor ideology goes here,
the white supremacy ideology of William Shockley and how that is
reflected in the bad decisions made in developing the modern microchip.
evils of clocks, connection to number worship and monotheism, sparse
desert of the processor, evil separations between matter and information
and energy
