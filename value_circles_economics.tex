\subsection{Moving Beyond Money}\label{moving-beyond-money}

Money is a failure. I have gone into some detail about this in the
earlier chapters and I will attempt to not belabor it here. How, exactly
does it fail, though? I say it's wrong to add up human value using
numbers, but what exactly is the mechanism by which this causes so much
misery? There are many mechanisms, but one that I think is worth
starting this chapter off with is its dissipative nature. As money flows
through our society it dissipates. At each stage from the banks out
through all the people working and selling things and paying each other
to when it comes back to the bankers, money is ``lost'' along the way,
usually to various parasitic organisms like landlords and online payment
gateways that add no value but simply take from the system.

So one feature I believe we need to look for in a value system beyond
money is that it should be additive rather than dissipative: rather than
transactions making value decrease, as they do for both money and the
physical goods capitalist claim as ``property'', we want value to
\emph{increase} as transactions happen. This goes against everything
capitalism stands for, as it doesn't work with currency based on integer
numbers, mining or human misery, but that's the point. The fact that
there is ``not enough money'' to do all the many things that need doing
in our society is partly because most of us do not have the ability to
make it. We can get it from someone else, but ultimately they had to get
it from someone who got it from someone who got it from a banker and of
course all those transactions paid the dissipations tax to the various
parasitic rent collectors. We can't just sit down, think really hard,
make a thing, talk to some people and \emph{make} something of value. We
can make a thing then sell it for money, but that is no longer a closed
system. If there are just the two of us, alone in the woods we cannot
create value in the money based system without also involving a banker.
Any system of value that we use to replace money must have the ability
to grow in value and scale in a closed system, without any need to
communicate with bankers, governments or others of their kind.

It is also worth at least mentioning the so-called ``alternatives'' to
money in the form of electronic currency like Bit Coin and the ``time
dollar'' or other currencies based on selling hours of your life away.
These are all still money. They use numbers to represent value, and
address almost none of the underlying problems with modern money.

In the special case of Bit Coin the creators have actually managed to
build perhaps the only money system worse than central bank debt
currency. They have replaced central bankers with members of the
technocratic priesthood who answer to no one and are some of the most
unpleasant and anti social people in our society. And they have replaced
an inflationary system with a deflationary one, making currency ever
more scarce as time goes on. I wish I were creative enough to have
invented Bit Coin as a sort of counter example of technocratic
capitalism gone mad, must like the Schroedinger's cat story used to show
the absurdity of quantum superposition(which still appears to accurately
describe the world). It would have been a great thought experiment to
show in a series of amusing anecdotes just what a horrible idea this is,
but alas, this is not fiction and we all have to deal with Bit Coin
people for the foreseeable future.

\subsection{Infinite and Infinitesimal
Value}\label{infinite-and-infinitesimal-value}

When you say you want to move beyond money for keeping track of value,
the capitalist will typically turn purple and start spitting about how
absurd the very idea is of doing anything like this. They have been
trained to do this, and it's a part of the immune system the capitalist
machine has built up over the centuries. However, they know perfectly
well that values without any numerical equivalent are quite common in
and essential for our society in even its current form. Our society and
indeed all societies have concepts of value outside finite number. Both
``priceless'' and ``worthless'' items are quite common in any system.

That which is called ``worthless'' by capitalists are often the feeds we
will use as ``Trash'' in Trash Magic, infinite supplies of things given
zero numerical value by the capitalists. Returning to the example from a
earlier chapter, the dog turd on the side of a street is an example of
something viewed as ``worthless'' or ``trash'' by the capitalists. To
take this example farther, what would it look like to attempt to apply
capitalist ``economics'' to the dog turd? Over some days it will be
digested by insects, bacteria and fungi. Once some atoms from the turd
have been consumed by a fly, does the fly own them? Who owns the fly? Or
perhaps the fly is a liability, whose is that? If the fly is eaten by a
bat who then uses those calories to also eat the mosquito which was
going to give me malaria surely the fly is now an asset, but whose?
Mine? Or the bat? But who owns the bat? And if I return to the turd a
couple weeks later and it's gone, did it ``depreciate'' to use the
jargon of accountants? Depreciating assets can generally be written off
as a discount on your taxes in most countries. How would I do that for
the dog turd? Perhaps if the turd was on land that I owned and I had a
numerical tally of the molecular wealth in the turd I could then
calculate how fast the flies are taking away this great wealth and
somehow turn this into numbers and then a tax write-off. But surely the
molecular wealth in the bellies of the well-fed flies are now an
appreciating asset. Is that then to be taxed? Who owns the flies? It's
all just nonsense! The vast universe around us of ``worthless'' things
proves the extreme limitation of the capitalist worldview to even
basically describe our environment.

But what about the ``priceless''? This is also an extremely familiar
concept in every capitalist society and is also one in which their
methods of assessing value completely and catastrophically fail.

\subsubsection{outline:}\label{outline}

\begin{itemize}
\tightlist
\item
  More on what is wrong with money, briefly, additive vs.~dissipative as
  key issue, need to free money from control, any control, bitcoin
  solves nothing, barter solves nothing, capitalist money diagrams and
  more of that, federal reserve debt, slavery reparations
\item
  value based on not numbers, but on ART, concept of ``priceless'' and
  ``worthless'' and ``trash'' and magic
\item
  oral traditions, tales and lore, folklore as a holder of value
\item
  the feed, a powerful metaphor
\item
  circles and other geometry as symbols of value
\item
  transmission of physical artifacts: couriers and
  thing-streams(skylines, rivers, various wandering drones)
\end{itemize}

\subsection{Magic Tales and Magic
Lore}\label{magic-tales-and-magic-lore}

I'm changing the name of the post capitalist system from ``value
circles'' to something closer to ``magic stories''. ``Value'' is a
problematic word since it sounds too much like an equivalent of the
capitalist economic system. I think it invites too many annoying
capitalism questions, and points the wrong way. And ``circles'' was
maybe a bit of a nonsense word in the way I was using it. What the hell
does that even mean? So I'm dropping both words.

One possibility I've been kicking around is ``narrative'', but that
sounds too impersonal, and too technical. I think ``magic'' sends the
right message since it is so clearly a turnoff word to engineers. One
thing that is clearly emerging here is that engineering as a discipline
and engineers as a culture are my enemy here. They are an important part
of the immune system of the Capitalist Monster that we are all parts of.
It is no accident that engineers like to tear down everything anyone
says all the time, being among the worst ``mansplainer's'' on the
planet. This is because that's where the rubber meets the road for
capitalism protecting its interests. The asshole male engineer who
always tells you you're wrong and nothing will work is really
capitalism's immune system, and in that context their behavior makes
perfect sense.

Possibly ``tale'' is better than stories. ``Magic Tales'' could also
then be abbreviated to ``tales'', which inevitably it would be. After
all, when one says ``tale'' instead of ``story'', does that not tend to
imply Magic? I think it does. Maybe where this should all end up with is
``tale''.

TALE: this is really remarkably easy to understand, and removes the hard
anti engineer bias of ``magic'' with something more flexible. Every
thing has a tale, every person has a tale, they are all woven together.
Perhaps Lore should also be used, and there is a lore data channel and a
tale data channel. Lore is how to build a thing, where to get more,
technological data. Tales are the story this specific thing.

Lore and Tales shall replace economics, and this becomes easy to
understand and explain without getting mired in pointless discussions
about capitalism. Because telling stories and lore has always existed
outside of capitalism, and can co-exist from the beginning. I'm pretty
sure HTML5 is the format, at least for now.

This is part of how Trash Magic can drive wedges into global capitalist
culture: we inhabit the strange corners of the current capitalist regime
where property-based values have not totally extinguished all goodness.
These include in the water, where property law is more flexible, and in
folklore as opposed to capitalist media, where IP has not taken over the
culture and suppressed creativity.

Note: I'm increasingly thinking that both the tale and lore should be
mostly oral. There will be graphical constructions to show how to do
stuff, but I think the combination of oral and graphical might be
superior to the written word for this, because I want the doing of
things to spread, rather than the empty talk. You have to be face to
face to really teach someone this stuff, so why not keep it offline with
exchanged files not on a techfuck server? It won't slow down
transmission of real stuff so why not? It will slow down the spread of
people talking about our ideas but doing nothing, but that's a good
thing. I want 100\% participation: you either walk away or you actively
participate. Reading blogs and even doing instructibles in your basement
in your house is not that. Those ``makers'' are of no use to this
movement except as a source of some materials and instructions in our
early days. Images and oral transmission is something I can actually do.
And this book. Fuck Github.

\subsubsection{Currency Diagrams}\label{currency-diagrams}

This will be pictures not words.

This is now how I see ``the system''. A circle of debt and power links
all people with business and finance to be deployed as needed to support
the military industrial complex. I no longer believe in the words
``money'' or ``government''. These are both fictions. There is only
debt, power, and the military industrial complex. All of this exists to
use fire to turn earth into debt and power and complete the cycle.
Denominating that debt by numbers which have power unto themselves
without the whole cycle is a unspoken State Religion adopted by all
modern states and corporations.

This is what value should look like:

People do labor using industrial ``waste'' of the old system until it's
all cycled through the new system, using ambient energy which comes
originally from the sun, and the living ecosystem that is supported by
and supports that cycle in circles of value. Circles can be formed large
and small, and involve trust between members of the circle which is
initially fixed and which has a finite lifetime. Circles can have any of
many different possible rules and structures, can live for a long time
or very briefly, etc. They might have as part of their interior various
physical artifacts or not, or various mathematical artifacts or not.

Circles may intersect in nodes which can have their own sets of rules.
The level of complexity of the infinitely expanding system of value
circles and nodes and networks has no serious theoretical limit. I
imagine that the amount of data required to denote a value circle is
always going to be small, even with some fairly verbose ASCII formatted
text about background, stories and rules etc. Media might be needed
which could take up a lot more space, but that should all be linked to
from the core value circle object.

\subsubsection{Creation of Value}\label{creation-of-value}

Suppose I have a motor I have built, and you have need for a motor.
Suppose I have built 1000 motors so I can easily spare a couple for your
robot. You need a robot, I can give you a robot, so of course I give you
a robot. Together, we have created value in the world in this
transaction because you having a robot and me having 999 robots is much
more useful than me having 1000 robots and you having zero robots. MUCH
more!

Right now we have two choices: we can just call it a gift, hand you the
robot, and I can feel like a nice guy. Or I can demand some ``money''
for it, after which we say I ``sold'' the robot to you. I put scare
quotes around these words, as I often do, to denote that I'm about to
reject the assumptions of these terms.

When people say ``money'' they mean debt from the Federal reserve bank,
or some other central bank. That debt has value because it is backed by
the military might of the United States, which accepts that debt in its
collection of taxes. But this is fucked. Why should we need debt from
some military backed bank in order to do this clearly positive
transaction? Surely just doing this adds value to society and we should
be able to denote that without federal reserve debt. But there is not
necessarily any motivation for anyone to make that possible.

So what is the alternative? It seems that the most common alternative is
the Marx-influenced concept of the time dollar. A local currency can be
created based on hours of labor which can be exchanged through a
community without any government involvement, taxes, or any banks. But
that is of no help for our robot transaction. My robots were built by
robots and took no labor. When you get the robot, it will do labor so
you don't have to. By carrying out a transaction that saves labor, we're
decreasing the value available in the system according to Marxian labor
theory of value. Anything that makes life easier creates deflation in a
labor based currency, which users of federal reserve debt can attest to
the horror of.

I propose that a usable way to communicate value outside of bank debt
will involve the ability of people carrying out a transaction to simply
create a marker of the value they mutually created. I also propose that
fancy math will not be the basis of this. Especially fancy math backed
by faith in libertarian neck beard fucks(you know what currency I'm
talking about). It will be based on trust. Trust of the people involved
in the transaction, which moves like a bubble through the untrusted mass
of society.

I propose that one way to do this is for a transaction to be a chapter
in a story, and that that story caries the value. So it works like this:
I give you a robot. We write a very short story about this, why we did
it why it was a good idea, why the robot is cool, etc. Short, to the
point, with some details. Now, I can take this story down to the coffee
shop and say ``hey, man, can I have some coffee, I gave someone a robot
today!'' They say ``yeah, you can have coffee here for the next week or
so for a robot, sure. That's the next chapter of the story. They pass
that along to their milk supplier, who adds another chapter and sends it
to the fence post company, who takes the longer story to an affiliate
coop out in the country who is part of our network, who delivers a much
more substantial wood processing robot machine. A real monster. And so
on.

It's not a fully formed system, but I don't think a good system ever
really will be. It's worth a try, better than nothing, better than
federal reserve debt.

\subsubsection{More on Value Circles}\label{more-on-value-circles}

Another element of the value circle currency concepts is myths. Myths,
legends, narratives, call it what you want. One way to create shared
trust between members of a value circle is to have shared culture, or
folklore.

Do people believe these to be actually true? Maybe. Maybe it doesn't
matter. My view is that existing money already has a weird religious
belief built in of the most dangerous possible kind: that which people
don't even acknowledge IS a belief. The entire world view created by the
monetary system where everyone has to exchange central bank debt is not
related to physical, social, or biological reality. It's a artificial
creation which harms most of the people who without having a choice or
even understanding that they made the choice are forced to live their
lives by it.

One way to combat what is essentially a very conservative religion is to
form a belief system outside it, making the transition from the money
belief system to a new one more explicit than just ``losing faith'' in
money which does not force the concept of money to be treated as a
religion.

What would be an example? I think initially they would tend to fall into
two categories: fan fiction and religion. One of the easiest ways to
build a mythology of a value circle is to do something like base things
off of Star Wars or Supernatural or something. It helps when people know
a thing well enough to have a shared reference easily right from the
start. For people who already have some sort of religion, building a
trust network based on that both formally and informally is an obvious
way to get started. Of course other values would be shared by a value
circle, including technically specific elements like ``meters of 24 AWG
copper magnet wire'', but on top of the specific parts, I believe having
something less quantitative and more personal is useful. More on this
later, this ongoing stack of aspects to the Value Circle.

\subsubsection{On Money and Additive
Value}\label{on-money-and-additive-value}

I hate money, and also love it, and that is typical of people in our
civilization. I've thought a lot about all that over the last year of my
total personal disillusionment with capitalism. I'm definitely against
most of how our ``economy'' works, and definitely in favor of something
else, but it's hard to even know where to start with all that. One habit
I've acquired over the last few months of reading and thinking about
anti-capitalism is replacing the word ``money'' in my mind with
``federal reserve debt''. That's literally what it is, and constantly
reminding myself of that helps me to think clearly about the world
around me and what to do about it.

One thing that I hate about money that I want to raise here is that it
is dissipative. When a transaction occurs, one party transfers their
federal reserve debt to another in exchange for some more real good or
service. That transfer has all kinds of losses in it. First of all, in
the money system the most value that can possibly exist after the
transaction is the amount you started with. Until another party is
brought in, in a single transaction, the amount of federal reserve debt
always goes down, just as the amount of entropy always goes up in
chemistry.

Looking at a system like this it's clear that the best way to accumulate
federal reserve debt is to be the dissipation. One way to do this is
literally to take something from the transaction, which is what paypal
and banks and credit cards and the rest of the finance industry do.
Another is to make money off taxes, as the military industrial complex
does. And a third is to be a middle man in the information channel from
seller to buyer, by being in the advertising/marketing industry. And
indeed I argue that these three types of accumulation are the main power
lines in our society: military, finance and marketing. Plenty of power
and wealth accumulators are all three or some combination, but I argue
that most power in our society is based on these three pillars because
they are the optimal means to accumulate federal reserve debt.
Everything else loses to these dissipations and eventually feeds someone
in one of these three pillars more than your little project possibly
ever can accumulate.

It is not so much my goal here to attack the concept of central bank
debt, taxes, etc. as to think about how to get outside this to add and
exchange value without that system. What I argue is that a transaction
should add a note of value to both sides, not just one, and that it
should require no value on a balance sheet before the transaction. This
second part is extremely critical. One of the crippling problems of our
current system is that it prevents anyone from being self sufficient,
ever. If some group wants to exchange goods and services in a closed
economy they need to first get federal reserve debt from the outside in
order to even have units of currency with which to work. Add dissipation
to that, and eventually they'll always be more and more dependent over
time on the outside world, and be forced to participate in global
capitalism. A system that addresses these problems must allow parties to
agree to do a thing, do it, and create from nothing the value that can
be further passed along to the rest of society. Another critical flaw in
the money system is the negative value of work. We assume that in any
work transaction there is a winner and a loser. E.g. at a gym everyone
has to either pay or get paid, it's assumed that the coaches are losing
something and the athletes are gaining, so they are on opposite sides of
a neutral or net-negative(with rent and taxes etc) transaction. But
surely the coaches also gain? Are they not also athletes? And the
athletes are working just as hard, why is their work somehow
``opposite''?

All this is cleared up by the additive currency concept. Here a
transaction creates a value pair, with half taken by each party. Thus
when a personal trainer meets with an athlete, they each walk away with
a unit of value equal to one times the value of that transaction. Let's
now go back to my motor factory supply chain. An urban scavenger rolls
up on their bike with a big bin of copper wire, and we each record that
that was of value and changed hands. They then take that value token to
the local coffee shop who pours them coffee and both sides get the
coffee transaction token. The coffee shop buys a coffee grinding machine
using one of our motors from one of our customer factories, more tokens
are generated on both sides. The coffee shop takes this new hoard of
tokens and pays their workers, and this payment also generates value on
both sides, further accumulating the wealth of the coffee shop who is a
major pillar in all this. The grinder factory trades with us for motors
in bulk, and some bulk material transaction value is again created on
both sides of the sheet.

There is a strong analogy between this system and the so-called h-index
used in academia. The h index is designed to create a measure of the
success of an academic career based on the combination of two factors:
how many times has someone published and how much are those publications
cited. The idea is to avoid valuing either the one paper that gets 1000
citations or the author who publishes 100 papers a year none of which
are ever read. Authors who both publish often and get frequently cited
are, on average, going to be the biggest contributors to value in the
field. For better or worse, h-index ends up having real value that can
get turned into federal reserve debt by having an impact on hiring and
promotions of academics. It's not a perfect system by any means and is
widely abused by departments but I think it's an interesting proof of
principle that this idea can be useful.

A missing part of all this is a proposed implementation strategy. How
should the value be accounted for? I could think of a lot of ways to do
it but I want to make the point that I think this is much less important
and difficult than a lot of techno nerds want you to believe. Any store
of value, whether it's paypal, cash, or credit card debt is basically
based on trust. Sure, there are anti-counterfeiting measures on bills
and encryption on online transactions. But for the most part these
systems work because most people can be trusted most of the time. If
everyone really were out to steal and cheat, encryption would be nowhere
near enough to save it, and it would collapse instantly. All this works
because the VAST majority of people would rather do something useful
than go into the illegal bill printing business or credit card theft.
One way I think it could be done is with an archive of stories. Some
kind of shared electronic narrative that includes all the transactions
in the network. This is not great for doing illegal shit or avoiding
government surveillance, and that is a problem in some ways. But not in
the long run because it forces people to push back against the
government controls a lot harder and faster and also because that stuff
can always still happen with federal reserve debt, alternate and more
anonymous systems, etc. Clearly there will be others for whom this
doesn't work. But I believe that a story- database-based system of value
can work for some people. And if it works for anyone it's instantly
extremely powerful because it will grow exponentially and naturally find
the people who can benefit from it most. Is it taxable? Probably not. If
we do things ``for free'' meaning no federal reserve debt is exchanged
at all, what is there to tax? Surely not vast, unencrypted databases of
anecdotes and poems describing the actions of millions of people.

And it's not even really necessarily a threat to the government tax
system, I think. Part of how our system is as broken as it is is how
differently it serves those who control the pillars of power from the
rest of us. The fact is that the capitalist overlords, governments,
military machines etc. don't really need the vast majority of us to
exist. Our demands for food, medical care, housing etc. are mostly an
inconvenience to them. An economy like this might take what looks like
potential tax revenue out of the system, but it also takes an incredibly
vast load of social welfare spending off of the existing system, since
that kind of value is so much better created in the additive value
system. One more point is that I don't consider bitcoin to be in any way
relevant to fixing the money system. With any form of currency, you have
to ask ``who do I trust when I place trust in this?'' I have lots of
criticisms of the central banks, and the federal reserve in particular.
But given the choice between central bankers and some neck beard fedora
software montherfuckers, I'll take the central bankers any day of the
week. Because the demographic in charge of bitcoin is, in my view, the
single least trustworthy group that exists in our society. Also,
building deflation into a currency is so bizarrely pathological it's not
even worth looking at. Bitcoin isn't money, neck beards are not
revolutionaries, it's time to move on.

So where do we start? I think I want to build a supply chain out of
trash, and then just try the database of stories method and see what
happens. Having a supply chain that is clearly of value will give me the
leverage to start a thing like this. So probably another year or two are
required, and hopefully by then I'll be more comfortable with python and
will be able to build a prototype software system to start this off.

\subsubsection{Free Feed of the Value
Circles}\label{free-feed-of-the-value-circles}

People love their feeds. Facebook, Twitter, Yammer, news feeds, tumblr
feeds, text message feeds, push notification feeds. It has proven to be
a very widely liked format for a person to see the passage of time of a
community of people. Value circles should include this feed concept.
Working on your stuff will add media content from your device which gets
added to the main value circle database and then fed into various users'
feeds based on their filter choices. I think if you are not trying to
make money that this whole thing can be much simpler than the existing
software and that Facebook etc can be replaced. However, it's also
possible that the best first implementation of this will be to do it in
a existing commercial system like Facebook. This is obviously dangerous.
Dealing with companies like that can have legal problems, control
problems, and limitations on what you can do practically. It's not
ideal, but it's something to consider. And the free feed that circulates
and shares data to a web browser that can be loaded on the pi zero
tablets of the trash wizards is a project that should be worked on
immediately. Probably tools already exist that can be adapted for this
task. This is worth some detail in the first book, it's not physics,
it's code and that's faster to deploy.

\subsubsection{Courier system}\label{courier-system}

I need a courier system to facilitate movement of goods as well as
skills without shipping costs

\subsubsection{Also value trees}\label{also-value-trees}

Is the circle the best geometrical metaphor? Sometimes. But I should
generalize the concept of geometric metaphors to include the spiral, the
pentagram, the tree, the fractal, the sphere, the torus etc. etc. trees
have roots and leaves and branches and a trunk and sap and sun and soil
and water.

\subsubsection{Trash Wizard Interaction with Capitalist
Economics}\label{trash-wizard-interaction-with-capitalist-economics}

We all have to live somehow, and most of us can't just sink beneath the
waves and escape capitalism overnight. We need money to pay rent to live
in a city where our loved ones are, to pay for medicine, get around on
capitalist transit etc. How do we do this ethically without just selling
out?

Here are the rules for Trash Wizard capitalist interactions:

\begin{enumerate}
\def\labelenumi{\arabic{enumi}.}
\tightlist
\item
  Try not to buy raw materials and other peoples labor, salvage trash
  and rely on mutual aid for free, and build your own stuff
\item
  To make money, selling labor is best, then selling stuff made from
  trash, always try to avoid labor and materials arbitrage: don't buy
  stuff then sell it don't pay people then re-sell their labor.\\
\item
  Do not sell misery, try to only work on things that are fun, while
  they stay fun, each product or service should be an ADVENTURE.
\end{enumerate}

We create. As long as our fed debt money comes from our labor and trash,
we will always have a net gain of capitalist currency into our system,
making hte sign of the dderviavie right. sell one off art pieces for
high prices. All my pieces should be unique.

\subsubsection{Specifically The Post Capitalist Sex
Industry}\label{specifically-the-post-capitalist-sex-industry}

Let's imagine a sex industry, with the word industry used, and inside
the capitalist system but without actual capitalism. What do I mean by
that? I mean that people who work make money but that they neither work
for a capitalist nor function in that role themselves. By ``capitalist''
I mean someone who pays one price for something and then sells it at
another price, whether that is wage labor or materials or outsourced
services. Usually they use some economy of scale to allow themselves a
monopoly enforced by the usual forces of armed capital. Exchanging
things of value for currency is not capitalism in that sense, as long as
you don't have to spend money for it.

I will now sketch out a ``business model'' for an anti capitalist sex
industry collective using Trash Magic.

First of all, how do you want work in this system to work? Really what
you want is to get paid about \$2000 for each sex event, to do it with a
bunch of your hot porn star friends who are into cool stuff and live in
your cool neighborhood, and to make art together and also build
something that useful, combining art and beauty.

So let's say that the goal here is to have about 6 people get together
for a queer techno art orgy and each make about 2 grand. That means you
need to make a total of about 12 grand. And you want to build a set of
artifacts which you sell for about 50 to 100 bucks, which a lot of
people will shell out for if they dig what you're doing. That means you
need to make, sell, and ship about 300 units.

So this orgy has to have the right set of robots and other machines so
that 6 people can make 300 units of an art-tech artifact in some
reasonable time. Suppose this whole thing takes a relaxed afternoon to
do, with all your stuff already set up in your Trash Magic Coven Space.
if we divide the number of units by the number of people it's 50 units
per person for the whole event. If each person has about 2 hours of full
participation in them before they get bored or tired, that's about 25
units per person per hour, or one unit every couple minutes for each
person. That's pretty efficient! Very doable with automation, but really
quite efficient and will require serious process optimization, which the
orgy process can help make happen.

Ok, so suppose that after your leisurely drugs/industrial
metal/manufacturing/art orgy you have made, and I'm going to boost the
numbers here, 500 units. You put those in an inventory file, which you
only give to people you put on a vetted customer list. Each unit
contains physical media with video and other data of the orgy for
pornographic purposes, and is a fully functional Trash Magic Stick. The
inventory system starts from scratch with each orgy, where each customer
has a one time customer ID number randomly generated and never used
again. Customer and performer alike then line up times and places in the
database to meet usually in public places, generally with several
customers for one performer and place, with performers always traveling
in pairs for safety to exchange the physical ``art piece'' for 50-100
dollars cash. Thus each transaction is simply money for art, in small
cash amounts(generally exempt from sales tax). All transactions will be
local, and performers will courier the inventory by bike around town.
The number of customers is about 300-500 so for 6 performers each one
can plan on about 3 distribution events each with about 20 customers. So
it's a second afternoon, on top of the first, and with a good number of
people for the events, easily each person could make 2-5k for a couple
days of work. If you did that about every two weeks, you could make
50-100k/year, enough to live a comfortable life where you spend 80\% of
your days doing other free art and science and the remaining 20\% having
orgies with your hot art/tech friends.

Many of us will end up doing serious R\&D during that other 80\% of the
time, enabling our factory to make more and more advanced stuff. After a
few years of R\&D I could imagine the customer getting a portable
medical imaging machine and some synthetic insulin with legit quality
control for their 50 bucks cash.

But what about the current clients, the horny old men with tons of cash,
the sexually disappointed owners of capital? They have to pony up much
more than cash if they expect to get what they want. They want some kind
of elaborate pro domme fantasy? Ok, sure, but they're not going to hand
over a stack of federal reserve notes, they're going to give you 80
tonnes of shipping space on their next freighter out of hong kong and 10
tons of rare earth magnets for your motor production. If they want to
use their position as controllers of the means of production to get sex
from younger, more attractive, and less broken people, that's all fine
but they're going to actually hand over those means of production not
some bank debt funny money.

Examples of what clients might hand over would be vast quantities of raw
metal, permission to work on private land without rent, and a lot of
long haul transport. Given that some captains of industry actually know
something about industry, this might even not be that bad for
performers, especially if people were well matched up in interests.

Also note that all this will be art. As fast as we'll be moving on the
assembly line to both have sex and guide the rumbles of robots, we'll
all also be working on the artistic process of shaping how the products
actually look, as well as putting on a performance that will be in the
final media.

In the beginning, obviously my fantasies of MRI machines and nano drugs
is not realistic, so what will we actually make first? The Trash Magic
Stick will be the first product. This is a simple artifact about the
size of a large walking stick, which can do many tasks that require an
electric motor. It can be modified by the user to have a powerful sex
toy, a mobile robot, a simple manipulator for 3d fabrication, a water
pump, a evaporative cooling unit, and several other useful machines. It
runs off of ambient energy as detailed in my book, usually flowing water
on site.

Yes, that's another thing, the physical site location. Ideally orgies
will take place in an amphibious environment of floating trash boats in
really nice weather, so that there is water power readily available both
to generate electric current for electronics and to do work and to
source water and other chemicals into the various industrial products.
So the Spring Orgy will be huge in this system. From the thaw of the
creeks through summer might just be one continuous orgy, followed by
just swimming and writing all summer and sleeping all fall. Again, if we
have the machines and robots we need, this is possible RIGHT NOW, under
capitalism, living in a nice fancy apartment in a big city and drinking
expensive coffee.

Oh, and how do we do that without getting the government coming after us
for tax evasion? By paying our taxes. Part of the infrastructure that
should exist here is a automatic LLC formation and dissolution as needed
for series of orgy/fabrication events. A company bank account will be
formed, money moved into it, taken out for taxes and distributed as
dividends to all the equal co-owners, then dissolved. So we'll actually
all be paying a very high tax rate. But who cares? We won't be putting
any money in, hiring anyone or working for anyone.

And this can also be scaled up quite a bit as high end pro doms fully
deploy their network of clients to start getting their hands on larger
quantities of raw metal and other critical process materials. We want to
avoid any materials getting mined on our behalf, because that will make
this technology non-free. However, if material is diverted \emph{out} of
the capitalist system in exchange for pro dom and other sex services
that is a net positive.

Lastly I want to consider a use case of this model that's scaled up a
bit, to say 100 people. This is something that could be done on a trash
flotilla surrounded by yachts in the waters outside LA during a sex
industry convention. 100 people come together and make, now with
considerable economies of scale and sophistication, 500 units each over
a week of work, with insane epic orgies of every possible variety. So
that's 50,000 units. And at this point each thing is really powerful,
can do significant high tech shit the capitalists can't do, like
universal medical imaging. So the performers sell them for 100 dollars
each, easily, making a total of 5 million dollars revenue for the
project(and gross = net because we spend zero money). If you divide 5
million dollars by 100 people, you get \$50,000 per person, enough to
live a decent life on for the whole year without any other work, even
after paying a high tax rate(which we should not have to do since this
will be capital gains if we have a decent accountant).

Does that sounds like unreasonably high production rates? I don't think
so. Current manufacturing technology already has huge economies of
scale. If those same scale economies are used with the same physical
basis of our technology but where everything is made from trash during
orgies, and we keep the value, this is really a pretty modest production
rate. With that many people involved, we'll have access to a ton of raw
materials, land, computer infrastructure, machines of various kinds etc.
And is the number of 50,000 customers for this scenario realistic? I
think it is. With mass communication via twitter and the like, it should
be possible to very easily and with no money spent to have the more
marketing oriented performers connect up with that many people in a
large metro area like LA or NYC, and then for each performer to
personally distribute 500 units in sets of 50 10 times by bike around
town. Serious work, but doable, and in just a few days if all the
infrastructure is in place.

With larger economies of scale and more high tech products, it will be
possible to have the products sell for quite a bit more, as well.
Something very high tech that costs many 10's of thousands of dollars(I
really want to build a fucking MRI machine,really want to) might be
built this way, making it still easy to sell the thing for 500 dollars
in this model, making the take home for each performer more like
\$250,000, meaning they only have to do an epic sex manufacturing art
science orgy like every few years if they want.

So this is how I propose the sex industry should work within capitalism.
It is a system where all performers are paid the same, make
\$50-250k/year for working a few weeks a year maximum, only ever have
sex with their hot porn star friends, and don't have to worry about cops
or the IRS because no laws are being broken and taxes are being paid.
Legally, all performers are professional artists, part of a series of
art-selling LLCs selling their own work built from trash.

How the fuck do we actually do this, you ask? That's my job: to build
the robots and write the instructions so that you can do it just by
watching some tutorial videos and downloading a bunch of free
instruction files. Give me another couple years and I think I can get
this working, but I think for very small numbers of units and people I
can get something working by fall of this year: making sex toys I mostly
already know how to make, which just need a bit of work. Art, though,
all art. Oh, yeah and free MRIs and the most powerful vibrators ever
built, running on off-grid power. Boom!

Also note that all this is Trash Magic. So these are magic witch orgies,
using all sorts of Trash Witchery, with associated Magic rituals
involving the process of assimilating trash into the world of magic.
