\subsection{Moving Beyond Money}\label{moving-beyond-money}

Money is a failure. I have gone into some detail about this in the
earlier chapters and I will attempt to not belabor it here. How, exactly
does it fail, though? I say it's wrong to add up human value using
numbers, but what exactly is the mechanism by which this causes so much
misery? There are many mechanisms, but one that I think is worth
starting this chapter off with is its dissipative nature. As money flows
through our society it dissipates. At each stage from the banks out
through all the people working and selling things and paying each other
to when it comes back to the bankers, money is ``lost'' along the way,
usually to various parasitic organisms like landlords and online payment
gateways that add no value but simply take from the system.

So one feature I believe we need to look for in a value system beyond
money is that it should be additive rather than dissipative: rather than
transactions making value decrease, as they do for both money and the
physical goods capitalist claim as ``property'', we want value to
\emph{increase} as transactions happen. This goes against everything
capitalism stands for, as it doesn't work with currency based on integer
numbers, mining or human misery, but that's the point. The fact that
there is ``not enough money'' to do all the many things that need doing
in our society is partly because most of us do not have the ability to
make it. We can get it from someone else, but ultimately they had to get
it from someone who got it from someone who got it from a banker and of
course all those transactions paid the dissipations tax to the various
parasitic rent collectors. We can't just sit down, think really hard,
make a thing, talk to some people and \emph{make} something of value. We
can make a thing then sell it for money, but that is no longer a closed
system. If there are just the two of us, alone in the woods we cannot
create value in the money based system without also involving a banker.
Any system of value that we use to replace money must have the ability
to grow in value and scale in a closed system, without any need to
communicate with bankers, governments or others of their kind.

It is also worth at least mentioning the so-called ``alternatives'' to
money in the form of electronic currency like Bit Coin and the ``time
dollar'' or other currencies based on selling hours of your life away.
These are all still money. They use numbers to represent value, and
address almost none of the underlying problems with modern money.

In the special case of Bit Coin the creators have actually managed to
build perhaps the only money system worse than central bank debt
currency. They have replaced central bankers with members of the
technocratic priesthood who answer to no one and are some of the most
unpleasant and anti social people in our society. And they have replaced
an inflationary system with a deflationary one, making currency ever
more scarce as time goes on. I wish I were creative enough to have
invented Bit Coin as a sort of counter example of technocratic
capitalism gone mad, much like the Schroedinger's cat story used to show
the absurdity of quantum superposition(which still appears to accurately
describe the world). It would have been a great thought experiment to
show in a series of amusing anecdotes just what a horrible idea this is,
but alas, this is not fiction and we all have to deal with Bit Coin
people for the foreseeable future.

\subsection{Infinite and Infinitesimal
Value}\label{infinite-and-infinitesimal-value}

When you say you want to move beyond money for keeping track of value,
the capitalist will typically turn purple and start spitting about how
absurd the very idea is of doing anything like this. They have been
trained to do this, and it's a part of the immune system the capitalist
machine has built up over the centuries. However, they know perfectly
well that values without any numerical equivalent are quite common in
and essential for our society in even its current form. Our society and
indeed all societies have concepts of value outside finite number. Both
``priceless'' and ``worthless'' items are quite common in any system.

That which is called ``worthless'' by capitalists are often the feeds we
will use as ``Trash'' in Trash Magic, infinite supplies of things given
zero numerical value by the capitalists. Returning to the example from a
earlier chapter, the dog turd on the side of a street is an example of
something viewed as ``worthless'' or ``trash'' by the capitalists. To
take this example farther, what would it look like to attempt to apply
capitalist ``economics'' to the dog turd? Over some days it will be
digested by insects, bacteria and fungi. Once some atoms from the turd
have been consumed by a fly, does the fly own them? Who owns the fly? Or
perhaps the fly is a liability, whose is that? If the fly is eaten by a
bat who then uses those calories to also eat the mosquito which was
going to give me malaria surely the fly is now an asset, but whose?
Mine? Or the bat? But who owns the bat? And if I return to the turd a
couple weeks later and it's gone, did it ``depreciate'' to use the
jargon of accountants? Depreciating assets can generally be written off
as a discount on your taxes in most countries. How would I do that for
the dog turd? Perhaps if the turd was on land that I owned and I had a
numerical tally of the molecular wealth in the turd I could then
calculate how fast the flies are taking away this great wealth and
somehow turn this into numbers and then a tax write-off. But surely the
molecular wealth in the bellies of the well-fed flies are now an
appreciating asset. Is that then to be taxed? Who owns the flies? It's
all just nonsense! The vast universe around us of ``worthless'' things
proves the extreme limitation of the capitalist worldview to even
basically describe our environment.

But what about the ``priceless''? This is also an extremely familiar
concept in every capitalist society and is also one in which their
methods of assessing value completely and catastrophically fail. When it
applies to an individual this is usually called ``sentimental value'',
and applies to well-loved personal things. This might be a t-shirt which
was purchased for very little and is too worn to still wear but which
was worn on some long journey. More often the most valuable personal
treasures are those which were given to us by others. We also use the
concept of priceless to refer to those things with shared cultural
value. Perhaps a capitalist can put a numerical price tag on things like
the various stone temples of our different human societies from
centuries past, but we all know that is not the real value. When a
ancient temple valued by the local government at some arbitrary number
of units of bank debt is destroyed no one in the world would dare say
that this is really the same as that much bank debt being destroyed.
It's not the same, because there are values in these cultural artifacts
which cannot \emph{ever} be added up using numbers.

An example of priceless value familiar to some students of American
popular culture is from the film Pulp Fiction from the 1990s. A
character played by the spectacular Christopher Walken presents a watch
to the son of his dead friend, and in one of the greatest monologs
recorded on film explains the priceless nature of the watch in the form
of a story. The story is not about the watch itself but about what
happened to the watch--it's not really about the thing but the people.
The watch ends up representing a four generation story of a male family
tradition of warrior values. Within that culture it has truly infinite
value. All this is shown as a flashback for a main character, to explain
why he is willing to lose everything including his own life to save the
watch. None of that is to tell time, to store value for retirement or to
hoard metal. The value of the watch is \emph{entirely} based on human
values and cannot possibly be translated to finite numbers.

I would propose a system of values where we all have our own personal
Christopher Walken telling the best possible stories about our things,
which give value the capitalists cannot possibly add up with numbers.
There are many ways to do this. In the end what all of them have in
common is that they lead to a value system that takes more from the
study of folklore than from the study of numbers. So the structure of a
post capitalist value system will take its basic shape from folklore.

Just as folklore is incredibly varied, from wood carvings of ancestors
to riddles and jokes to songs and epic poems, the tools we use to
express value in industrial product should be as diverse as any other
kind of lore.

\subsection{Tales and Lore}\label{tales-and-lore}

This discussion of ``lore'' brings me to two jargon terms I will
introduce for Trash Magic industry/art: tales and lore. Things in
general should have both a tale and lore if they are to exist in a value
system after capitalism. A tale is exactly like the story above with
Christopher Walken and the gold watch. Part of the power of that story
is that it keeps going: the viewers of the film see yet another dramatic
component of the story, which presumably will be passed on to yet
another generation eventually. In this case it is entirely oral,
although of course in this case it is a tale within a tale since we are
watching a film which tells a made up story of an oral tradition. I
think there are many ways to do this, the most obvious being purely
oral, although the universal ability of people to both upload videos to
youtube and watch them there using the now-ubiquitous smartphone argues
that online video archives might be a way to combine oral tradition with
simple and free recordings. Books, poems, carved murals, paintings,
decorative rope work and songs all might also be a part of the tale. But
whatever we end up doing, the point is everything should have some sort
of tale. That tale might be ``this was made in part of a giant assembly
run of 10,000 units in some factory that actually sold them for money,''
but that's still a more complete tale than we usually get under
capitalism. And it's just the beginning! Over time, a thing will have
stuff happen to it, and all that should get added to the tale. The value
of that tale will clearly go up as it gets passed around.

Unlike capitalism, which encourages hoarding, it's clear that giving
things away to whoever will make the best use of them vastly increases
their value over simple hoarding. ``This thing was in my closet for 10
years'' is clearly a much less powerful story than ``this thing was
given to someone who hitchhiked across the continent with it'' or
something else where the use and need are more substantial.

So that is the tale, but what is lore? Lore is the knowledge that is
part of building a thing, and all the associated culture that goes into
that. In our present society, lore already always exists around things
that are made, whether it is the lumberjack's knowledge of how to safely
fell a tree, the programmers knowledge of how to use some key hardware
driver, or the factory worker's skill on a drill press. However, as with
the tale of creation, this lore is not passed on to the user in
capitalism because it generally separates us so rigidly into separate
categories: maker, owner, engineer, artist, user, worker, etc.

No more! We need lore that can really be passed on, the way operating a
car or changing lightbulbs is now passed on in capitalist society.
Again, this is probably easiest to start with face to face oral
tradition, where you directly teach in a hands on way the future user
how to make a thing. But just as with the tale, it can be a problem to
carry it through a completely oral tradition because of all the times we
want to pass goods on to others without having to travel, so we need
ways to record both tales and lore. Again, the recording of videos and
uploading to youtube is a good place to start. Also for larger
artifacts, passing along some type of flash drive or other cheap and
small memory which holds all the files for the designs as well as videos
and images can help physically transmit tales and lore without the
Internet or travel.

Ultimately we should be developing our own memory system which can
record analog video in the molecular structure of grown minerals of
various kinds after full nanotech is working. Before that we should be
growing interface mineral structures to found flash memory from salvaged
trash which can store stuff, and before that off the shelf flash cards
can be decoratively incorporated into various artifacts, with graphical
artistic instructions which direct the future user to the media.

Note again that capitalists are already using these ideas in their own
ways. It has always been common in history for various types of
organized crime(including both government and various lords of capital)
to patronize the arts to create ``priceless'' cultural
artifacts(e.g.~the whole renaissance), then to set up a market system
where ``stolen'' works of art are held in illegal warehouses and traded
around as value-holding items outside of the banker system. This is well
documented in the literature of art theft history, and proves there is
precedent for art being used as a type of currency outside of government
and banker control. If the art market has always played a subversive
role in capitalism, surely the ability to create infinite streams of art
can even more so.

What I hope I have shown with this section is that the ideas required to
build value into industrial products without money are already familiar.
All we need is to look at what is already there and thoughtfully apply
it to the tools we have and we can immediately do interesting work
outside capitalism. What, specifically does that look like? My example
will be the things I actually build in the process of creating this
volume. The fact that they are the co-products of writing this book is
the initial tale of them. The lore will be the documentation I put in
later parts of this book about how to make them. And as I give it to you
as a gift(which I will do as many times as I can as well as I can) that
act of gift will be another step of the tale. And the lore will carry
directly from me to you both through reading this book and talking about
it and also through the instruction that I hope to deliver to all who
seek it over the next few years, showing everyone how to build on and
grow and make more of all this. If it does grow, the value of these
tales will continue to rise for \emph{all} of us, from me as creator to
you as participant and onward to your successors. Just as capitalism
tends to lead to a repeated pattern of they pyramid where the lowest and
largest level is always crushed by the higher ones, we hope to build a
future that also has repeating patterns, but those patterns are ones of
abundance.

It's ok to start small! Take meaningless junk, paint it in a way that
tells a story, glue stuff to it to make it useful, then give it to a
friend, tell the story, and pass that on. If they do the same back to
you but with something else, you have now \emph{both} created a greater
value than you started with, and with no bank or government intervention
and no numbers.

Also note that this is not barter. Economists love to use barter as a
club to beat non-money-worshippers over the head with. But as David
Graeber tells us, this is largely made up for that purpose--barter has
always primarily been something used with untrusted people in
essentially capitalist ventures. It's still number-worship, still
numerical values used for everything because you simply have to find
direct equivalents for everything. You do not need Christopher walken
monologs for barter--clearly a problem. Like bit coin and the time
dollar, I cast the ideas of barter aside as capitalist propaganda and
religious nonsense, to be mostly ignored as we try to build a better
world.

\subsection{The Feed}\label{the-feed}

People love their feeds! As horrible as they often are, the various
social media feeds that dominate modern life are fantastically powerful
tools. In Facebook, Twitter, Youtube, Instagram, and probably a hundred
other sites I don't know about, users have the ability to quickly scroll
up and down though a timeline that mixes the output timeline of many
very different entities. Often the timeline you see will mix local news,
foreign news, personal announcements from friends, artistic output of
various artists, promotions for other artists, weather data, and
numerous other types of useful and (potentially)interesting information.

Given that these feeds are easier and easier to build with modern
software, are generally free and are well known and liked already, I
think they should play a part in how we pass lore and tales along.
Perhaps things you make can each have a tumblr feed, and you pass the
password along to the next person who gets it, they keep adding to the
tale and lore both on the feed, then pass it along in the same way when
they're done. Or a youtube account, with google used to do following, or
various ways of using Twitter. I'm not sure, but what I propose is that
we keep in mind this basic concept(independent of implementation) and
then just try it and see what works. There will be many solutions found
by many people over time.

Should those feeds be encrypted? Maybe. That is up to you, I want to be
completely open about this, and would hope that some will go a fully
open route and others will build something with very strong physical
encryption. Many paths for many people should be a constant goal in this
value system, and that includes how the feeds are transmitted.

\subsection{Geometry of Value}\label{geometry-of-value}

In earlier versions of this manifesto I got sucked in to drawing all
sorts of strange diagrams of how I see the geometry of capitalist money.
You can draw pyramids of many kinds that represent how the top extract
from the bottom in capitalism. But who cares? They mostly don't need
geometry since their number worship regards numbers higher than shapes.

What is much more interesting than adding to the anti-capitalist crank
literature is trying to build up a geometry of value \emph{outside} of
capitalism. The first thing that comes to mind for this is the circle.
As many people have now pointed out, if we want to make our value
systems more sustainable and more like Nature the circle is a commonly
recurring shape. Nature is full of circular processes, and often
objects, like droplets of water, form in a spherical shape, generalizing
the circle. As with the ``feed'' mentioned above, none of this is
literal, but then taking math literally is what got the number
worshipers in trouble. It's an image we put in our minds when designing
processes. It's helpful to think of a circle when building a mental
model for how economics might work with tales and lore outside
capitalism.

Other geometric ideas can have a powerful resonance in how we decide to
structure value without number. Those include various fractal patterns
such as the spiral or the fern-like structure. Also any of the numerous
polyhedra that mean things to people, including the various oddly shaped
dice used for various role playing games can be useful. The helix has
become a universal symbol of life since the discovery of the structure
of DNA, and is also used for screws, a fairly universal simple machine.
A helix can be a great geometric metaphor for a stable relationship
between a pair of entities that are intertwined and move around relative
to each other in a simple way. And finally the crystal lattice can be
powerful, albeit with the hazard that it has too much of a number
worshipping flavor.

The tree is a powerful image in just about every possible belief system,
including the scientific understanding of the living world. Of course
trees of all kinds and every possible part of their world should be a
part of our value system imagery.

\subsection{Trash Magic Conveyor}\label{trash-magic-conveyor}

One of the key elements of the capitalist monopoly on control of goods
is all the parts of the supply chain after the factory. This includes
shipping, warehousing, distribution, display in stores, transport from
stores to homes, disposal, and possibly a used market.

We need this as well, of course. How do we get things from where they
are made to where they are used? What if I just want to make things and
send them on? Or make nothing and just grab things?

There need to be conveyor systems that move goods along on their own
without help. This can be lazy rivers, skylines, or air tubes. In the
first version of this I imagine them being very localized, making a sort
of Trash Magic equivalent of the conveyor belt sushi concept on a creek.
I am imagining that one of us will sit by the creek, building up the
conveyor and making some things, then put the things on the conveyor and
move on, allowing future passers by to take what they want and continue
the story(increasing each things value as this makes an interesting
story). If even a small fraction of those people decide to make more and
put them back on the conveyor it is easy to see how this can lead to an
exponentially growing economic system outside the system of capitalism.
It's art, craft, and outdoor amusement. Also some science and technology
and industry--but not capitalism, as no numbers are used at all in the
``transaction'' of making a thing and setting it on the conveyor or
grabbing it off the conveyor, watching some videos, and making some
more.

Our system of goods distribution can also involve simple self powered
drones that dumbly move across the landscape powering themselves,
getting fixed by passerby, moving goods where they go, and navigating by
some simple formula(head north, head downstream, follow the cold, follow
the wind, find a city etc.) This sounds hard at this point in time of
course, as we are just getting started. The simplest possible free
distribution system I know of is the message in a bottle: you write a
message on a piece of paper, put it in a sealed bottle(so the air makes
it float) and throw it out to sea. I can personally attest that doing
this can result in strangers from far away writing a physical letter in
response. Surely innovative technology/art can also be delivered this
way, and across national and language and culture barriers as well!
Large numbers of us who put these bottle with beautiful and useful art
into the water at one end of a major ocean current(like the Japanese
current that pushes water from Japan to Alaska bringing bamboo to all
Alaskan kids to play with) and the artifacts will be received in large
numbers by people in a far away land with a totally different culture
and language and nationality. If you take this route, clarity is key! A
picture is worth a thousand words, and videos should not require verbal
understanding if it's possible to communicate purely by demonstration.
If you know the language of the reader, or even have an educated guess
get a translation of at least enough to tell them how to translate the
rest(mark the language you use in many other languages).

The final and possibly most important leg of our basic supply line will
be the human courier. Nothing is more personal than personal delivery!
Let's try to make things as much as possible tailored to people we meet
in real life, possibly in the course of making more things, and give
them directly as gifts to those people. This makes tales and lore of a
very detailed nature possible, both oral and with various media, and
with clear one on one hands on instruction in the actual lore required
to make the thing. The hands on courier method will almost certainly by
the most powerful vector for spreading exponentially since it will have
a higher quality of transmission of the ability to make things. Of
course I don't have to start all this. You can just come up with your
own interpretation of what all this means, make some useful art, and go
out and teach and give it. Now. Or wait until we come to you, because we
will eventually!

\subsection{Trash Wizard Interaction with Capitalist
Economics}\label{trash-wizard-interaction-with-capitalist-economics}

Finally, we all have to live somehow, and most of us can't just sink
beneath the waves and escape capitalism overnight. We need money to pay
rent to live in a city where our loved ones are, to pay for medicine,
get around on capitalist transit etc. How do we do this ethically
without just selling out?

Here are some proposed rules for Trash Wizard capitalist interactions:

\begin{enumerate}
\def\labelenumi{\arabic{enumi}.}
\tightlist
\item
  Try not to buy raw materials and other peoples labor; salvage trash
  and rely on mutual aid for free, and build your own stuff
\item
  To make money, selling labor is best, then selling stuff made from
  trash, always try to avoid labor and materials arbitrage: don't buy
  stuff then sell it don't pay people then re-sell their labor.\\
\item
  Do not sell misery, try to only work on things that are fun, while
  they stay fun, each product or service should be an ADVENTURE.
\end{enumerate}

We create. As long as our fed debt money comes from our labor and trash,
we will always have a net gain of capitalist currency into our system,
making the direction of change right to allow us to survive best in the
current system.
