\section{A Rumble of Robots}\label{a-rumble-of-robots}

The collective noun for robots is ``a rumble of robots.'' I'm not sure
where I heard this, I think on of my friends may have made it up, but
it's so perfect it's too good not to use. So I want to talk about
rumbles of robots. In particular the difference between robots used for
consumption and for production.

Amazon is in the process of building robot based infrastructure for
delivery. This is fundamentally a consumption driven project. The main
initial figure of merit in the growth of their network will be coverage:
the more potential consumers are covered, the better. This will mean
that it is optimal for robots to be as far as possible from other
robots. But how does this picture change for production?

Rumbles of robots are very common on the production side of things.
Those who produce cars and computers and the like often have rumbles of
robots, with humans just as technicians who run the machines. Much like
a cow hand or shepherd, I think there should be a name for those who
herd rumbles of robots: rumbler. So the trash wizard is also a rumbler.
And the trash wizard stick is like the shepherd's crook: a tablet that
drives a IoT network that consists of your rumble of robots.

That is what seizing the means of production is really all about. It's
not about seizing an existing factory, which will be based on existing
methods, or about building a primitive system that can't compete. It's
about building rumbles of robots which can reproduce themselves by
harvesting free materials to make more, and then rumbling them around to
build what else you need.

Key elements of the trash wizards robot rumble are mobility and
versatility. They will run off of locally harvested energy, and be
programmed to gather energy as needed as well as materials. They should
scale in that the robots you need for a 10 bot rumble are not so
different from a single roninbot or a 1000 bot rumble. They should be
able to reproduce from found materials and forage for those materials
with some simple guidance from the rumbler. That is the plan.

\section{Another Demo that Must be Built: Octahedral Ball
Drone}\label{another-demo-that-must-be-built-octahedral-ball-drone}

The octahedral ball drone is a octahedron made of three intersecting
sticks, with a flexible joint. Each of the 6 ends has a 2 degree of
freedom +/- pair of control coils and magnets with some significant
amount of effective length change. Drivers can use the natural dynamics
of the flexing rods to make efficient rolling motion. I imagine making
the first one based on a LiPo battery and Raspberry Pi, with wooden
sticks held with a rubber central ball that the sticks can get pushed
into. I imagine the sticks being a total of about 12 to 15 inches long
total, so that each pod is about 6-8 inches from the center, and can
flex by about 2-3 inches, or 20\% to 50\% of total. The amount of play
should be just barely more than enough to get the full sphere effect for
maximally efficient rolling. I imagine for the very first version all
coils are driven by dumb feedback and it's an art piece that rolls
around in all directions chaotically until the battery dies. Then all
the boards get tied together to a central Raspberry Pi, which gets onto
the net for control by a remote console. Then the code task gets solved
for the simplest possible rolling motion with directional bias. Then
some kind of simple chase algorithm is written so that it can track down
a target that emits some kind of signal. This demo could launch a whole
robotics project which can be a main support column for The Project.

\section{Wind Desert Drift Robots}\label{wind-desert-drift-robots}

Rolling robots with windmills, they roll, then gather wind electricity
into a capacitor, roll again, and repeat. They can go for hundreds of
miles with no intervention. The instinct to go a certain direction based
on navigating off of the sun is programmed into the physical hardware.
After some long time, maybe many years, the machine calls for help,
eventually someone finds it and follows the instructions for repair and
improvement. With generation after generation editing and helping the
thing exist, it can exist for hundreds of years, slowly cleaning up
wasted sacrafice zones of the old capitalist world.

There are so many machines to build like this! Machines that comb the
ocean for contaminants, using waves go get energy to move around and
sort and grab stuff, potentially floating around for years before being
found based on a data stream that pulses out periodically, and
eventually another type of robot tending robot can grab it, extract the
materials it's gathered, and bring it to a floating factory robot
rumble. This kind of robot is important for the ecosystem of the jungle
city in the ocean-inundated coastal post apocalypse.

Free robots like this are a rational response to the fact that the
existing system has created sacrafice zones. These sacrafice zones have
negative economic value in the old system, making them freely available
to be absorbed into the anarchist industrial infrastructure. This is
key: in order to avoid getting crushed by the forces of the old system
too early our movement must exist in the fringes of the current system,
where the old ways have created land of negative value. The very fact
that land can have negative value, that this is a concept that people
accept, should be yet another red flag that assignment of numerical
values to real human values is a morally bankrupt act.

This should always be the goal of free technology if it wants to grow
exponentially without a lot of resistance: the input must be things
deemed of ``negative value'' by the old system. Unlike most projects in
capitalism which constantly drain everyone involved more and more over
time, creating generation after generation of institutional burnout.

\section{Building an Ent}\label{building-an-ent}

The fractal mater reactor should be alive. Trees, bushes, grass, etc.
can grow all around it, with roots going into various fractal channels
which provide nutreince. These liquid spaces can have various animals
and fungi and microorganisms, creating a whole ecosystem. Imagine an
island built up of such mater, the size of a small building, covered
with trees. Ambient energy is used to slowly build up and discharge
electrical energy to operate philosophy engines which slowly walk the
whole thing across the landscape. With little or even no human
intervention, this limbering living giant might spend decades crawling
up and down hills scouring for junk cars, which it turns into a ever
growing robot rumble that it can give away to any passing humans for
free at any time. Building this kind of thing in the ocean can be
incredibly powerful. Whole floating islands filled with fractal reactor
technology can wander the high seas, with the humans all underwater in
bubbles to ride out storms, picking up storm energy and sea junk, and
building a every larger floating city deep out in the ocean. This
aquatic fractal techno city could exist even in a dead world of violent
storms and acid oceans and extreme heat. That's part of why I hate the
liberal vision of fighting global warming by trying to make everything
into Denmark. It won't actually work, and then you still have a
breakable society. Sure our society is destroying itself, but so what?
Let's build a million new ones, not just one. And let's expect a future
not of competent bureaucrats carefully tabulating the giant World
Spreadsheet so everyone can live the life of a middle class urban Sweede
but a world where ice ages and supervolcanoes and nuclear wars and
devastating earthquakes all happen and where we fucking roll with it and
have adventures.

Start by building and documenting some actual robots, make a rumble,
rumble the rumble with the trash wizard stick.

\begin{enumerate}
\def\labelenumi{\arabic{enumi}.}
\tightlist
\item
  hopper dumb robot
\item
  hopper with a brain using raspberry pi for browser to pi to board to
  motor robot control
\item
  roller robot
\item
  rolling ball robot
\item
  hopper rumble
\item
  make a mobile robot that can use a hot tool to rework thermo plastics
\item
  floating robot that can re work thermo plastic(loop back to these on
  fractal reactor chapter)
\end{enumerate}

\section{add a whole section on the plastic bottle welding fabrication
process}\label{add-a-whole-section-on-the-plastic-bottle-welding-fabrication-process}

Robots will be built that can cut plastic bottles, as well as weld
plastic bottles together. These robots will be made from the same
process. ALso there will be flying robots that soar based on gossamer
bottle wing construction with a ability to flap. Also rolling robots
will be bottle based. Very simple!

\section{Free Robots for Free
Hardware}\label{free-robots-for-free-hardware}

\section{Rumbles of Robots}\label{rumbles-of-robots}

Add: littoral robot rumbles which can use tides and river currents to
generate electricity to propel themselves upstream. Can be amphimious,
use water to charge but land to move, can move with hopping, jumping,
walking, rolling, and slithering. Littoral trash cleanup robots are
fundable, can make a huge difference in cleanup of a waterway, and also
give us free source material for more building.

\section{Hopping, rolling, walking, running, jumping,
driving}\label{hopping-rolling-walking-running-jumping-driving}

\section{Free food with robots}\label{free-food-with-robots}

\section{Robots with different times scales, centuries of work, or hours
of
lifetime}\label{robots-with-different-times-scales-centuries-of-work-or-hours-of-lifetime}

\section{Robots Building Robots}\label{robots-building-robots}

\subsubsection{rumble jacks}\label{rumble-jacks}

Imagine the following drone/robot: magnets on duct tape with 2 drive
coils each are on the vertices of a octahedron made from a star of
sticks with the brain and energy storage in the middle. With the right
kind of intelligence, all the magnets can move in a coordinated way to
roll the whole ball like a jack in the game jacks. Each jack is aware of
the position of its neighbors, and together they make a rumble of robots
that act like a herd of sheep. Hence ``rumbleJack'' for the rumble of
jacks like robots.

This technology can be used for all sorts of long slow land cleaning
processes. Rather than try to maximize battery life, they will use
capacitors to store energy, and recharge the capacitors from ambient
energy. For a rumble of jacks in the prairie, the obvious source of
power is the wind. Ideally, the wind will be used to create energy which
will immediately go into directed propulsion. This might be slow since
it depends on gusts, but it can go on forever, so slowness becomes ok.
This is technology that you would deploy to spend 1000 years cleaning up
a sacrifice zone, where you want no outside energy or materials to be
needed at all and for the rumble to keep doing its work for hundreds of
years. Also, obviously, clearing of mine fields is a immediate
application. A rumble of tire-sized octahedra could potentially roll
themselves at 10's of miles per hour, keeping up with a car or truck and
making it possible for the rumble to proceed in a mob ahead of a motor
vehicle, taking out IED's in real time. The rumble could end up in a
convoy geometry, stretched out over the length of the road, doing recon
ahead and tracking behind to see what's happening after a convoy passes.
In these applications it probably makes sense for the source of power to
be the trucks or cars in the human/freight convoy, with individuals in
the rumble cycling through the charging station and back out into the
rumble.

Going back to the plaines discussed above, this is a great tool for
agriculture. Even just gathering. A gathering rumble could go out and
gather roots and berries from the countryside in a quasi-cultivated
area. These roving balls could be picking up and dropping seeds as they
go, mapping where all the useful plants are, and also harvesting as they
go, taking wind sun and water as energy sources as needed, then spending
energy when it's available to do the work.

Another robot rumble I want to build that is closely related is the
slithering water robots. These use the usual magnet and coil arrangement
to create a slithering motion in buoyant objects, which can then
smoothly cut through the water. The fact that this has not been widely
deployed is totally insane: the same drive can be used in reverse to get
electrical power out of wave action. If the length of each robot is a
few wavelengths, the whole thing will be forced into a wave which can
create EMF as the magnets move, which can go into the storage
capacitors, then released to change slightly the nature of the
serpentine motion to direct the drone in a specific direction.

These can be incredibly powerful technology! The ocean can be a
fantastic source of raw materials for the trash wizards. Note that for
neutrally buoyant drones, this can serve to move them through the water
below the surface. One mode of operation might be to cruise a few meters
above the bottom of the ocean, scanning for stuff to salvage, then dive
and grab rocks to be negatively buoyant once a target is found. With
just barely negative buoyancy, the rumble can float just above the
target as they pick it apart. They then drop the weights, rise up,
inflate bags to float(everything is made from rubber, and reversible
air/vacuum/water pumps are in all things), and pull up and bring
material back to assembly centers, which can also be floating robot
rumble factories. With ocean currents and waves as an energy source, and
no hurry, these robots can work as slow as they have to, slowly making
more and more of themselves until they can have a global impact on ocean
cleanup.

The water based propulsion system also is very appealing for boats. I
want a boat that runs on wave action, wind, and tides, to grab energy as
it finds it, and then use it as needed to move toward a destination. I
can imagine this being just about kayak or canoe sized. I could also
imagine a freighter that is meters or even 10's of meters long. That
sounds small for a freighter, but imagine, again, that they're a huge
rumble that can be easily scaled up. This can be a freight swarm to move
materials across water.

One more thing I want to talk about relating to these technologies: the
serpentine gliding drone. This would be made out of very thin light
frame with large thin polymer wings to allow for gliding. Magnet-coil
drives are used to pulse the shape to optimize gliding, and to interact
with turbulence(linear wind won't work because we need relative motion
up in the air) to acquire energy,probably in the form of \emph{Very}
high voltages because that is most compatible with being high up.
Perhaps electric fields that exist due to natural weather can also be
used for various electrical things up high. A flying swarm that can
glide and gather energy and never have to land if the conditions are
right has many applications. Mapping to gain information on materials
for trash wizards to salvage is an obvious application. This type of
flying serpentine glider can also be used as hardware for data transfer
by flying flash memory. Donors like this are also a fantastic source of
much more granular weather and climate data than are now available. This
is useful in the long term for practical climate and weather science
studies to deal with mitigation during the coming storms of the next 200
years from climate change. But they also have use for short term weather
sensing tasks: tracking down storms that can be used as energy sources
for very large scale projects.

What about scaling these robots way up in size and weight for use inside
storms? One could imagine giant metal gliders in massive rumbles of 10's
of thousands or maybe even millions of units, all ripping around in a
storm could over the ocean. These generate giant hydrogen-filled blimps
which then gather in a huge rumble to go turn back into useful work near
a settlement or floating factory.

The Anthropocene is here. Like it or not, it's here. For the next 1000
years our planet is going to be dominated by the actions we choose to
take as a civilization. If we stay on the track we're on, the atmosphere
and oceans heat up, massive desertification destroys wet ecosystems
while rising oceans eat most of our cities, and the oceans become a
toxic waste dump that cannot sustain life. If we do nothing that is
clearly what will happen. Or something worse involving nuclear
holocaust. Given these alternatives, what difference does it make how
drastically we change things in the sea, air, and land? The opportunity
to simply not let civilization get big enough to destroy the world has
long passed us by now.

So is it so wrong to imagine the whole landscape filled with these
lumbering rumbles of rolling, slithering, hopping, and gliding robots?
Is it wrong to let them reproduce with human help, but with very little
labor-time, allowing groups of people to build endlessly expanding
rumble spheres around the world to create a world of total abundance? I
say that it is not wrong. Maybe if there were a way to go back it would
be a hard choice to do something that disturbs the balance of nature
like this, but there isn't.

This is what the trash wizard wants to make possible in the world:
endless streams of material and data moving through the physical world
with robots made from trash, which encompass our whole human
environment. Maybe not the whole world, but enough of it. A world of
abundance using the rumble sphere and value circles could exist outside
of the states and corporations. It does not need land, just someplace to
move to--it is all mobile by default. The trash wizards build the needed
expertise up and document it and teach it so that any group of people
can create this kind of culture anywhere, specific to their individual
cultural needs and the available resources in whatever geographical area
they're in.
